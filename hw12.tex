\documentclass[10pt,a4paper]{article}
\usepackage{amsmath, amssymb, amsthm}
%加這個就可以設定字體
\usepackage{fontspec}
\usepackage{xkeyval} %MikTeX 2.9 版本相容有誤, 以此修正
%使用xeCJK,其他的還有CJK或是xCJK
\usepackage{xeCJK}
%設定英文字型,不設的話就會使用預設的字型
%\setmainfont{Times New Roman}
%設定中英文的字型
%字型的設定可以使用系統內的字型,而不用像以前一樣另外安裝
%\setCJKmainfont{文泉驛微米黑}
\setCJKmainfont{WenQuanYi Micro Hei}
\setCJKfamilyfont{lh}{LiHei Pro}
\newcommand{\LiHei}{\CJKfamily{lh}}
%中文自動換行
\XeTeXlinebreaklocale "zh"
%文字的彈性間距
\XeTeXlinebreakskip = 0pt plus 1pt
%設定段落之間的距離
\setlength{\parskip}{0.3cm}
%設定行距
%\linespread{1.5}\selectfont
\usepackage{enumerate}

%先暫時用水泥字型,如果任何人看不下去就改吧XD
\usepackage[T1]{fontenc}
\usepackage{concmath}
%\usepackage{mathptmx} %Times

\usepackage{tabularx}

\newcounter{theProblemCounter}
\newtheorem{problem}[theProblemCounter]{Problem}

\begin{document}
\title{{\fontspec{Copperplate Gothic Bold}Geometry Homework 12}}
%\title{{\fontspec{Copperplate}Geometry Homework 11}}
\author{{\it{B96201044}} {\LiHei 黃上恩}, {\it{B98901182}} {\LiHei 時丕勳}, {\it{K0020100x}} {\LiHei 劉士瑋}}
\date{\today}
\maketitle

\newcommand{\bx}{\mathbb{X}}
\newcommand{\bfx}{\mathbf{x}}
\newcommand{\grad}{\textrm{grad }}
\newcommand{\sech}{\mbox{sech}}

%第三題
\setcounter{theProblemCounter}{2}
\begin{problem}[Ex p294 3.]
If $p$ is a point of a regular surface $S$, prove that \[ K(p) = \lim_{r\to 0}\frac{12}{\pi}\frac{\pi r^2-A}{r^4}, \] where $K(p)$ is the Gaussian curvature of $S$ at $p$, $r$ is the radius of a geodesic circle $S_r(p)$ centered in $p$, and $A$ is the area of the region bounded by $S_r(p)$.
\end{problem}

%第四題
\setcounter{theProblemCounter}{3}
\begin{problem}[Ex p295 4.]
Show that in a system of normal coordinates centered in $p$, all the Christoffel symbols are zero at $p$.
\end{problem}

%第五題
\setcounter{theProblemCounter}{4}
\begin{problem}[Ex p295 5.]
For which of the pair of surfaces given below does there exist a local isometry?
\begin{enumerate}
\item[(a)] Torus of revolution and cone.
\item[(b)] Cone and sphere.
\item[(c)] Cone and cylinder.
\end{enumerate}
\end{problem}

%第八題
\setcounter{theProblemCounter}{7}
\begin{problem}\hspace*{1em}
\begin{enumerate}
\item[(a)] 在半徑 $R$ 的球面上,計算 geodesic circle 的長度,並驗證 P292 課文中間 $K(p)$ 的公式。
\item[(b)] 用一樣的精神,檢驗 P294 3. 的公式。
\end{enumerate}
\end{problem}
\end{document}
