\documentclass[10pt,a4paper]{article}
\usepackage{amsmath, amssymb, amsthm}
%加這個就可以設定字體
\usepackage{fontspec}
\usepackage{xkeyval} %MikTeX 2.9 版本相容有誤, 以此修正
%使用xeCJK,其他的還有CJK或是xCJK
\usepackage{xeCJK}
%設定英文字型,不設的話就會使用預設的字型
%\setmainfont{Times New Roman}
%設定中英文的字型
%字型的設定可以使用系統內的字型,而不用像以前一樣另外安裝
%\setCJKmainfont{文泉驛微米黑}
\setCJKmainfont{WenQuanYi Micro Hei}
\setCJKfamilyfont{lh}{LiHei Pro}
\newcommand{\LiHei}{\CJKfamily{lh}}
%中文自動換行
\XeTeXlinebreaklocale "zh"
%文字的彈性間距
\XeTeXlinebreakskip = 0pt plus 1pt
%設定段落之間的距離
\setlength{\parskip}{0.3cm}
%設定行距
%\linespread{1.5}\selectfont
\usepackage{enumerate}

%先暫時用水泥字型,如果任何人看不下去就改吧XD
\usepackage[T1]{fontenc}
\usepackage{concmath}
%\usepackage{mathptmx} %Times

\usepackage{tabularx}

\newcounter{theProblemCounter}
\newtheorem{problem}[theProblemCounter]{Problem}

\begin{document}
\title{{\fontspec{Copperplate Gothic Bold}Geometry Homework 13}}
%\title{{\fontspec{Copperplate}Geometry Homework 13}}
\author{{\it{B96201044}} {\LiHei 黃上恩}, {\it{B98901182}} {\LiHei 時丕勳}, {\it{K0020100x}} {\LiHei 劉士瑋}}
\date{\today}
\maketitle

\newcommand{\bx}{\mathbb{X}}
\newcommand{\bfx}{\mathbf{x}}
\newcommand{\grad}{\textrm{grad }}
\newcommand{\sech}{\mbox{sech}}
\newcommand{\pr}[2]{\frac{\partial #1}{\partial #2}}
\newcommand{\prr}[3]{\frac{\partial^2 #1}{\partial #2\partial #3}}
\newcommand{\ip}[2]{\left\langle#1, #2\right\rangle}

%第四題
\setcounter{theProblemCounter}{3}
\begin{problem}
Helicoid $\bx(u, v)=(v\cos u, v\sin u, u)$, $\gamma(t)=\bx(t, 1)$, $p=\bx(0, 1)=(1,0,0)$, $V(0)=\gamma'(0)$ 求解平行向量場 $V(t)$ along $\gamma(t)$
\end{problem}

%第六題
\setcounter{theProblemCounter}{5}
\begin{problem}
如圖考慮一旋轉體上的緯圈 $\gamma$,已知其 generating curve(經線) 切線與中心軸夾角為 $\theta$。
\begin{enumerate}
\item[(a)] 求一向量沿 $\gamma$ 平行移動,繞一圈後與原向量的夾角 (不妨假設起始向量與緯圈同向)
\item[(b)] 將該 surface 放大或縮小,相對應問題的夾角有何變化
\item[(c)] 計算此緯圈之 $\oint_\gamma \kappa_g \mathrm{d}s$,值與 surface 的縮放有關嗎?
\end{enumerate}
\end{problem}

%第十題
\setcounter{theProblemCounter}{9}
\begin{problem}[Ex P282 4.]\hspace*{0em}
\begin{enumerate}
\item[(a)] Compute the Euler-Poincar\'e characteristic of (1) An ellipsoid. (2) The surface $S=\{(x, y, z)\in \mathbb{R}^3; x^2+y^10+z^6=1\}$.
\item[(b)] 如圖,將一圓盤的邊界如圖「黏」起來(也可以想成將對稱點「黏」起來),找一個三角分割,計算此 projective space 的 Euler characteristic。
\end{enumerate}
\end{problem}

%第十二題
\setcounter{theProblemCounter}{11}
\begin{problem}[Ex P283 6.]
Show that $(0,0)$ is an isolated singular point and compute the index at $(0,0)$ of the following vector fields in the plane:
\begin{enumerate}
\item[(a)] $v=(x, y)$.
\item[(b)] $v=(-x,y)$.
\item[(c)] $v=(x,-y)$.
\item[(d)] $v=(x^2-y^2,-2xy)$.
\item[(e)] $v=(x^3-3xy^2,y^3-3x^2y)$.
\end{enumerate}
\end{problem}

\end{document}
