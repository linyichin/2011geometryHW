\documentclass[10pt,a4paper]{article}
\usepackage{amsmath, amssymb, amsthm}
%加這個就可以設定字體
\usepackage{fontspec}
\usepackage{xkeyval} %MikTeX 2.9 版本相容有誤, 以此修正
%使用xeCJK,其他的還有CJK或是xCJK
\usepackage{xeCJK}
%設定英文字型,不設的話就會使用預設的字型
%\setmainfont{Times New Roman}
%設定中英文的字型
%字型的設定可以使用系統內的字型,而不用像以前一樣另外安裝
%\setCJKmainfont{文泉驛微米黑}
\setCJKmainfont{WenQuanYi Micro Hei}
\setCJKfamilyfont{lh}{LiHei Pro}
\newcommand{\LiHei}{\CJKfamily{lh}}
%中文自動換行
\XeTeXlinebreaklocale "zh"
%文字的彈性間距
\XeTeXlinebreakskip = 0pt plus 1pt
%設定段落之間的距離
\setlength{\parskip}{0.3cm}
%設定行距
%\linespread{1.5}\selectfont
\usepackage{enumerate}

%先暫時用水泥字型,如果任何人看不下去就改吧XD
\usepackage[T1]{fontenc}
\usepackage{concmath}
%\usepackage{mathptmx} %Times

\newcounter{theProblemCounter}
\newtheorem{problem}[theProblemCounter]{Problem}

\begin{document}
%\title{{\fontspec{Copperplate Gothic Bold}Geometry Homework 10}}
\title{{\fontspec{Copperplate}Geometry Homework 9}}
\author{{\it{B96201044}} {\LiHei 黃上恩}, {\it{B98901182}} {\LiHei 時丕勳}, {\it{K0020100x}} {\LiHei 劉士瑋}}
\date{\today}
\maketitle

\newcommand{\bx}{\mathbb{X}}
\newcommand{\bfx}{\mathbf{X}}
\newcommand{\grad}{\textrm{grad }}
\newcommand{\sech}{\mbox{sech}}
%\newcommand{\cosh}{\mbox{cosh}\ }
%\newcommand{\tanh}{\mbox{tanh}\ }
%\newcommand{\sinh}{\mbox{sinh}\ }

%第二題
\setcounter{theProblemCounter}{1}
\begin{problem} 若 $E=1, F=0, G=1, f=0$,假設再加入函數 $e, g$ 後是某 surface 的 1st\&2nd fundamental form。
\begin{enumerate}
\item[(a)] 說明 $e, g$ 中至少有一為 0
\item[(b)] 說明若 $e=g=0$ 則此曲面為平面
\item[(c)] 說明若 $e\ne 0$,則此曲面為特別的 ruled surface,並討論 $e$ 的意義。
\end{enumerate}
\end{problem}
\begin{proof}
\begin{enumerate}
\item[(a)]
Since $E=1, F=0, G=1$, we know that the surface has $K=0$, so $eg-f^2=0$. So $eg=0$, and one of $e$ and $g$ is zero.
\item[(b)]
If $e=g=0$, then
\begin{align*}
\bx_{uu}&=[1,1,1]\bx_u+[1,1,2]\bx_v+eN=0\\
\bx_{uv}&=[1,2,1]\bx_u+[1,2,2]\bx_v+fN=0\\
\bx_{vv}&=[2,2,1]\bx_u+[2,2,2]\bx_v+gN=0
\end{align*}
So $\bx_u$ and $\bx_v$ are constant, and the surface is a plane.
\item[(c)]
if $e\ne 0$, then $g=0$, and $\bx_{vu}=\bx_{vv}=0$, so $\bx_v$ is constant.\\
So $\bx(u,v)$ is a line when we fix $u$, thus $\bx$ is a ruled surface.\\
Let $\gamma(u)=\bx(u,0)$, then $\|\gamma'(u)\|=\|\bx_u\|=1$, so $u$ is arc-length parameter for $\gamma$.\\
$\gamma''(u)=\bx_{uu}(u,0)=eN$, so $\textrm{sign}(e)N$ is also the $n$ for $\gamma$, and $\gamma$ has curvature $|e|$.
\end{enumerate}
\end{proof}

%第四題
\setcounter{theProblemCounter}{3}
\begin{problem}[Ex p237 8.]
Compute the Cristoffel symbols for an open set of the plane
\begin{enumerate}
\item[(a)] In cartesian coordinates.
\item[(b)] In polar coordinates.
\end{enumerate}
Use the Gauss formula to compute $K$ in both cases.
\end{problem}
\begin{proof}
\begin{enumerate}
\item[(a)]
Let $\bx(u,v)=(u,v,0)$, then $\bx_{uu}=\bx_{uv}=\bx_{vv}=0$, so $\Gamma_{ij}^k=0 \forall i,j,k=1,2$.\\
So $R_{1212}=0$ and $K=0$.
\item[(b)]
Let $\bx(u,v)=(u\cos v,u\sin v,0)$, then:
\begin{align*}
\bx_u&=(\cos v,\sin v,0)\\
\bx_v&=(-u\sin v,u\cos v,0)\\
N&=(0,0,1)\\
E=1, F&=0, G=u^2\\
\bx_{uu}&=(0,0,0)\\
\bx_{uv}&=(-\sin v,\cos v,0)\\
\bx_{vv}&=(-u\cos v,-u\sin v,0)\\
\begin{bmatrix}
\Gamma_{11}^1 & \Gamma_{12}^1 & \Gamma_{22}^1\\
\Gamma_{11}^2 & \Gamma_{12}^2 & \Gamma_{22}^2\\
\end{bmatrix}&=
\begin{bmatrix}
1 & 0\\0 & u^2
\end{bmatrix}^{-1}
\begin{bmatrix}
0 & 0 & -u\\
0 & -u & 0\\
\end{bmatrix}\\
&=\begin{bmatrix}
0 & 0 & -u\\
0 & -\frac{1}{u} & 0\\
\end{bmatrix}
\end{align*}
\begin{align*}
R_{112}^2&=\Gamma_{11,2}^2-\Gamma_{12,1}^2+\Gamma_{11}^1\Gamma_{21}^2+\Gamma_{11}^2\Gamma_{22}^2-\Gamma_{12}^1\Gamma_{11}^2-\Gamma_{12}^2\Gamma_{12}^2\\
&=\frac{1}{u^2}-\frac{1}{u^2}=0\\
\rightarrow K&=\frac{R_{1212}}{EG}\\
&=\frac{R_{112}^2}{E}=0
\end{align*}
\end{enumerate}
\end{proof}

%第六題
\setcounter{theProblemCounter}{5}
\begin{problem}
有一 surface $\bfx(u, v)$,令 $\hat{\bfx}(u, v)=\lambda \bfx(u,v), \lambda > 0$。
\begin{enumerate}
\item[(a)] 討論 $\hat{\Gamma}^k_{ij}$ 和 $\Gamma^k_{ij}$ 的關係
\item[(b)] 從 Gauss equation(GTE) 討論 $\hat{K}$ 和 $K$ 的關係
\end{enumerate}
\end{problem}
\begin{proof}
\begin{enumerate}
\item[(a)]
\begin{align*}
\hat{g}_{ij}&=\langle\hat{\bfx}_i,\hat{\bfx}_j\rangle\\
&=\lambda^2\langle\bfx_i,\bfx_j\rangle\\
&=\lambda^2 g_{ij}\\
\rightarrow \hat{g}^{ij}=\frac{1}{\lambda^2}g^{ij}\\
\langle\hat{\bfx}_{ij},\hat{\bfx}_k\rangle&=\langle\lambda\bfx_{ij},\lambda \bfx_k\rangle\\
&=\lambda^2\langle\bfx_{ij},\bfx_k\rangle\\
\rightarrow \hat{\Gamma}_{ij}^k&=\hat{g}^{kl}\langle\hat{\bfx}_{ij},\hat{\bfx}_l\rangle\\
&=g^{kl}\langle\bfx_{ij},\bfx_l\rangle\\
&=\Gamma_{ij}^k
\end{align*}
\item[(b)]
\end{enumerate}
Since $\hat{\Gamma}_{ij}^k=\Gamma_{ij}^k$, $\hat{R}_{ijk}^l=R_{ijk}^l$.\\
\begin{align*}
\hat{R}_{imjk}&=\hat{g}_{ml}\hat{R}_{ijk}^l\\
&=\lambda^2 g_{ml}R_{ijk}^l\\
&=\lambda^2 R_{imjk}\\
\rightarrow \hat{K}&=\frac{\hat{R}_{1212}}{\hat{E}\hat{G}-\hat{F}^2}\\
&=\frac{1}{\lambda^2}\frac{R_{1212}}{EG-F^2}\\
&=\frac{1}{\lambda^2}K
\end{align*}
\end{proof}
%第九題
\setcounter{theProblemCounter}{8}
\begin{problem}
舉一個例子說明有可能 $F:M\to N$ 是 conformal map,且相應點 $K_M>0, K_N=0$ (想想曾經討論的例子)
\end{problem}
\begin{proof}
取 $M$ 為單位球 $x^2+y^2+z^2=1$, $N$ 為平面 $z=0$, 則顯然 $K_M>0, K_N=0$.\\
取 map $f: M\mapsto N$, $f(x,y,z)=(\frac{x}{1-z},\frac{y}{1-z},0)$ 為 stereographic projection.\\
因為若 $f(x,y,z)=(u,v,w)$, 則
\begin{align*}
du^2+dv^2+dw^2&=\left(\frac{(1-z)dx+x dz}{(1-z)^2}\right)^2+\left(\frac{(1-z)dy+y dz}{(1-z)^2}\right)^2\\
&=\frac{1}{(1-z)^4}\left((1-z)^2dx^2+(1-z)^2dy^2+(x^2+y^2)dz^2+2(xdx+ydy)(1-z)dz\right)\\
&=\frac{1}{(1-z)^4}\left((1-z)^2dx^2+(1-z)^2dy^2+(-z^2+1)dz^2+(-2zdz)(1-z)dz\right)\\
&=\frac{1}{(1-z)^4}\left((1-z)^2dx^2+(1-z)^2dy^2+(1-z)^2dz^2\right)\\
&=\frac{1}{(1-z)^2}\left(dx^2+dy^2+dz^2\right)\\
\end{align*}
So $f$ is a conformal mapping, but $K_M>0, K_N=0$.
\end{proof}
\end{document}
