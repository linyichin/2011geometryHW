\documentclass[10pt,a4paper]{article}
\usepackage{amsmath, amssymb, amsthm}
%加這個就可以設定字體
\usepackage{fontspec}
\usepackage{xkeyval} %MikTeX 2.9 版本相容有誤, 以此修正
%使用xeCJK,其他的還有CJK或是xCJK
\usepackage{xeCJK}
%設定英文字型,不設的話就會使用預設的字型
%\setmainfont{Times New Roman}
%設定中英文的字型
%字型的設定可以使用系統內的字型,而不用像以前一樣另外安裝
%\setCJKmainfont{文泉驛微米黑}
\setCJKmainfont{WenQuanYi Micro Hei}
\setCJKfamilyfont{lh}{LiHei Pro}
\newcommand{\LiHei}{\CJKfamily{lh}}
%中文自動換行
\XeTeXlinebreaklocale "zh"
%文字的彈性間距
\XeTeXlinebreakskip = 0pt plus 1pt
%設定段落之間的距離
\setlength{\parskip}{0.3cm}
%設定行距
%\linespread{1.5}\selectfont
\usepackage{enumerate}

%先暫時用水泥字型,如果任何人看不下去就改吧XD
\usepackage[T1]{fontenc}
\usepackage{concmath}
%\usepackage{mathptmx} %Times

\newcounter{theProblemCounter}
\newtheorem{problem}[theProblemCounter]{Problem}

\begin{document}
\title{{\fontspec{Copperplate Gothic Bold}Geometry Homework 10}}
%\title{{\fontspec{Copperplate}Geometry Homework 9}}
\author{{\it{B96201044}} {\LiHei 黃上恩}, {\it{B98901182}} {\LiHei 時丕勳}, {\it{K0020100x}} {\LiHei 劉士瑋}}
\date{\today}
\maketitle

\newcommand{\bx}{\mathbb{X}}
\newcommand{\bfx}{\mathbf{X}}
\newcommand{\grad}{\textrm{grad }}
\newcommand{\sech}{\mbox{sech}}
%\newcommand{\cosh}{\mbox{cosh}\ }
%\newcommand{\tanh}{\mbox{tanh}\ }
%\newcommand{\sinh}{\mbox{sinh}\ }

%第二題
\setcounter{theProblemCounter}{1}
\begin{problem} 若 $E=1, F=0, G=1, f=0$,假設再加入函數 $e, g$ 後是某 surface 的 1st\&2nd fundamental form。
\begin{enumerate}
\item[(a)] 說明 $e, g$ 中至少有一為 0
\item[(b)] 說明若 $e=g=0$ 則此曲面為平面
\item[(c)] 說明若 $e\ne 0$,則此曲面為特別的 ruled surface,並討論 $e$ 的意義。
\end{enumerate}
\end{problem}

\begin{proof}
\end{proof}

%第四題
\setcounter{theProblemCounter}{3}
\begin{problem}[Ex p237 8.]
Compute the Cristoffel symbols for an open set of the plane
\begin{enumerate}
\item[(a)] In cartesian coordinates.
\item[(b)] In polar coordinates.
\end{enumerate}
Use the Gauss formula to compute $K$ in both cases.
\end{problem}

%第六題
\setcounter{theProblemCounter}{5}
\begin{problem}
有一 surface $\bfx(u, v)$,令 $\hat{\bfx}(u, v)=\lambda \bfx(u,v), \lambda > 0$。
\begin{enumerate}
\item[(a)] 討論 $\hat{\Gamma}^k_{ij}$ 和 $\Gamma^k_{ij}$ 的關係
\item[(b)] 從 Gauss equation(GTE) 討論 $\hat{K}$ 和 $K$ 的關係
\end{enumerate}
\end{problem}

%第九題
\setcounter{theProblemCounter}{8}
\begin{problem}
舉一個例子說明有可能 $F:M\to N$ 是 conformal map,且相應點 $K_M>0, K_N=0$ (想想曾經討論的例子)
\end{problem}
\end{document}
