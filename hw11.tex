\documentclass[10pt,a4paper]{article}
\usepackage{amsmath, amssymb, amsthm}
%加這個就可以設定字體
\usepackage{fontspec}
\usepackage{xkeyval} %MikTeX 2.9 版本相容有誤, 以此修正
%使用xeCJK,其他的還有CJK或是xCJK
\usepackage{xeCJK}
%設定英文字型,不設的話就會使用預設的字型
%\setmainfont{Times New Roman}
%設定中英文的字型
%字型的設定可以使用系統內的字型,而不用像以前一樣另外安裝
%\setCJKmainfont{文泉驛微米黑}
\setCJKmainfont{WenQuanYi Micro Hei}
\setCJKfamilyfont{lh}{LiHei Pro}
\newcommand{\LiHei}{\CJKfamily{lh}}
%中文自動換行
\XeTeXlinebreaklocale "zh"
%文字的彈性間距
\XeTeXlinebreakskip = 0pt plus 1pt
%設定段落之間的距離
\setlength{\parskip}{0.3cm}
%設定行距
%\linespread{1.5}\selectfont
\usepackage{enumerate}

%先暫時用水泥字型,如果任何人看不下去就改吧XD
\usepackage[T1]{fontenc}
\usepackage{concmath}
%\usepackage{mathptmx} %Times

\usepackage{tabularx}

\newcounter{theProblemCounter}
\newtheorem{problem}[theProblemCounter]{Problem}

\begin{document}
\title{{\fontspec{Copperplate Gothic Bold}Geometry Homework 11}}
%\title{{\fontspec{Copperplate}Geometry Homework 11}}
\author{{\it{B96201044}} {\LiHei 黃上恩}, {\it{B98901182}} {\LiHei 時丕勳}, {\it{K0020100x}} {\LiHei 劉士瑋}}
\date{\today}
\maketitle

\newcommand{\bx}{\mathbb{X}}
\newcommand{\bfx}{\mathbf{x}}
\newcommand{\grad}{\textrm{grad }}
\newcommand{\sech}{\mbox{sech}}
%\newcommand{\cosh}{\mbox{cosh}\ }
%\newcommand{\tanh}{\mbox{tanh}\ }
%\newcommand{\sinh}{\mbox{sinh}\ }

%第四題
\setcounter{theProblemCounter}{3}
\begin{problem} Show that if all the geodesics of a connected surface are plane curves, then the surface is contained in a plane or a sphere.
\end{problem}

%第五題
\setcounter{theProblemCounter}{4}
\begin{problem} Let $\alpha: I\to S$ be a curve parametrized by arc length $s$, with nonzero curvature. Consider the parametrized surface
\[ \bfx(s, v) = \alpha(s)+vb(s), \ \ s\in I, -\epsilon<v<\epsilon, \epsilon > 0,\]
where $b$ is the binormal vector of $\alpha$. Prove that if $\epsilon$ is small, $\bfx(I\times (-\epsilon, \epsilon)) = S$ is a regular surface over which $\alpha(I)$ is geodesic. (thus, every curve is a geodesic on the surface generated by its binormals).
\end{problem}
\begin{proof}
\begin{align*}
\bfx_s&=\alpha'(s)+vb'(s)\\
&=t(s)+v\tau(s)n(s)\\
\bfx_v&=b(s)\\
\rightarrow \bfx_s\times\bfs_v&=-n(s)+v\tau(s)t(s)\\
&\neq 0
\end{align*}
So $\bfx$ is a regualr surface.\\
Since $\alpha''(s)=n(s)$, and at $\alpha(I)$, $N\parallel \bfx_s\times\bfs_v=-n(s)$. So $\alpha'(s)\parallel N$, and $\kappa_g=0$. So $\alpha(I)$ is geodesic.
\end{proof}

%第八題
\setcounter{theProblemCounter}{7}
\begin{problem}
用 (A) 表示在座標變換下不變、用 (B) 表示在 isometry 下不變 (保 $E, F, G$) 下的性質

\begin{tabular}{c|c|c|c|c|c|c|}
\cline{2-7}
 & line of curvature & geodesic & asymptotic curve & $\Gamma^{k}_{ij}$ & $H$ & $K$ \\
\cline{2-7}
(A) & & & & & & \\
\cline{2-7}
(B) & & & & & & \\
\cline{2-7}
\end{tabular}

%第九題
\setcounter{theProblemCounter}{8}
\begin{problem}
考慮 p221, p222 中 helicoid $Y$ 和 catenoid $X$ 的 parametrization。
\begin{enumerate}
\item[(a)] $X$ 中的 geodesics 相對應映到 $Y$ 中也是 geodesics 嗎?
\item[(b)] 已知 $X$ 的經線 ($u = $ const) 與 $v=0$ 都是 geodesics。描述他們在 $Y$ 中的對應曲線?他們都是 geodesics 嗎?
\end{enumerate}
\end{problem}

\end{problem}
\end{document}
