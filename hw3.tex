\documentclass[10pt,a4paper]{article}
\usepackage{amsmath, amssymb, amsthm}
%加這個就可以設定字體
\usepackage{fontspec}
\usepackage{xkeyval} %MikTeX 2.9 版本相容有誤, 以此修正
%使用xeCJK,其他的還有CJK或是xCJK
\usepackage{xeCJK}
%設定英文字型,不設的話就會使用預設的字型
%\setmainfont{Times New Roman}
%設定中英文的字型
%字型的設定可以使用系統內的字型,而不用像以前一樣另外安裝
\setCJKmainfont{文泉驛微米黑}
\setCJKfamilyfont{lh}{LiHei Pro}
\newcommand{\LiHei}{\CJKfamily{lh}}
%中文自動換行
\XeTeXlinebreaklocale "zh"
%文字的彈性間距
\XeTeXlinebreakskip = 0pt plus 1pt
%設定段落之間的距離
\setlength{\parskip}{0.3cm}
%設定行距
%\linespread{1.5}\selectfont

%先暫時用水泥字型,如果任何人看不下去就改吧XD
\usepackage[T1]{fontenc}
\usepackage{concmath}
%\usepackage{mathptmx} %Times

\newcounter{theProblemCounter}
\newtheorem{problem}[theProblemCounter]{Problem}

\begin{document}
\title{{\fontspec{Copperplate Gothic Bold}Geometry Homework 3}}
\author{{\it{B96201044}} {\LiHei 黃上恩}, {\it{B98901182}} {\LiHei 時丕勳}, {\it{K0020100x}} {\LiHei 劉士瑋}}
\date{\today}
\maketitle

%第三題
\setcounter{theProblemCounter}{2}
\begin{problem}[P26: 16]
Show that the knowledge of the vector function $n=n(s)$ (normal vector) of a curve $\alpha$, with nonzero torsion everywhere, determines the curvature $\kappa(s)$ and the torsion $\tau(s)$ of $\alpha$.  ($\vec{n}$ 能決定曲線嗎?說明題目錯誤並找反例。)
\end{problem}
\begin{proof}
Consider the helix $\alpha(s)=(a\cos{\frac{s}{\sqrt{a^2+b^2}}},a\sin{\frac{s}{\sqrt{a^2+b^2}}},{\frac{bs}{\sqrt{a^2+b^2}}})$\\
Then $n(s)=(-\cos{\frac{s}{\sqrt{a^2+b^2}}},-\sin{\frac{s}{\sqrt{a^2+b^2}}},0)$.\\
So if two helix has the same $a^2+b^2$ (e.g. $\alpha_1(s)=(\frac{1}{2}\cos{s},\frac{1}{2}\sin{s},\frac{\sqrt{3}}{2}s)$, $\alpha_2(s)=(\frac{\sqrt{3}}{2}\cos{s},\frac{\sqrt{3}}{2}\sin{s},\frac{1}{2}s)$), then they have same $n(s)$, but they're not the same curve (because $\kappa=\frac{a}{a^2+b^2}$, so they have different $\kappa$).
\end{proof}

%第四題
\setcounter{theProblemCounter}{3}
\begin{problem}[P26: 17,另一種描述 Helix 的方式]
In general, a curve $\alpha$ is called a \emph{helix} if the tangent lines of $\alpha$ make a constant angle with a fixed direction. Assume that $\tau(s)\ne 0$, $s\in I$, and prove that:
\begin{enumerate}
\item[(a)] $\alpha$ is a helix if and only if $\kappa/\tau=$constant.
\item[(b)] $\alpha$ is a helix if and only if the lines containing $N(s)$ and passing through $\alpha(s)$ are parallel to a fixed plane.
\item[(c)] $\alpha$ is a helix if and only if the lines containing $B(s)$ and passing through $\alpha(s)$ make a constant angle with a fixed direction.
\item[(d)] The curve\[
\alpha(s)=\left(\frac{a}{c}\int \sin\theta(s) ds, \frac{a}{c}\int \cos\theta(s) ds, \frac{b}{c} s
\right),
\]
where $c^2=a^2+b^2$, is a helix, and that $\kappa/\tau=a/b$.
\end{enumerate}
\end{problem}
\begin{proof}
WLOG, assume that $s$ is arc-length parameter, $V$ is the fixed direction. Let $T(s)=\alpha'(s)$, $T'(s)=\kappa N(s)$ and $B=T\times N$.
\begin{enumerate}
\item[]
\item[(a)]
\begin{align*}
\langle T,V\rangle&=C\\
\rightarrow \langle T',V\rangle&=0\\
&=\langle\kappa N,V\rangle\\
\rightarrow \langle N,V\rangle&=0\\
\rightarrow \langle N',V\rangle&=0\\
&=\langle-\kappa T-\tau B,V\rangle\\
&=-\kappa C-\tau \langle B,V\rangle\\
&=-\kappa C-\tau \langle T\times N,V\rangle\\
&=-\kappa C-\tau \langle V\times T,N\rangle\\
V&=\langle V,B\rangle B+\langle V,T\rangle T+\langle V,N\rangle N\\
&=\langle V,B\rangle B+C T\\
\langle V,B\rangle'&=\langle V,\tau N\rangle=0\\
\rightarrow \langle V,B\rangle&=constant\\
\rightarrow \langle V\times T,N\rangle&=\langle \left(\langle V,B\rangle B+C T\right)\times T,N\rangle\\
&=\langle \langle V,B\rangle B\times T,N\rangle\\
&=\langle V,B\rangle\\
\rightarrow 0&=-\kappa C-\tau \langle V\times T,N\rangle\\
&=-\kappa C-\tau\langle V,B\rangle\\
\rightarrow \kappa/\tau=-\frac{\langle V,B\rangle}{C}
\end{align*}
So $\kappa/\tau$ is constant.

Conversely, let $\kappa/\tau\equiv c$ be a constant. Define vector $V$ by $V(s) = T(s)-cB(s)$. We claim $V$ is a constant since $V'(s)=T'(s)-cB'(s)=\kappa(s) N(s) - c\tau(s) N(s) = (\kappa(s)-c\tau(s))N(s)=0$. Now $\left\langle V, T\right\rangle$ is constant because $\left\langle V, T\right\rangle'=\kappa\left\langle V,N\right\rangle = \kappa\left\langle T-cB, N\right\rangle = 0$. This implies $T$ make a constant angle with $V$.
\item[(b)]
$\left\langle V, T\right\rangle \equiv c$ implies $0=\left\langle V, T\right\rangle' = \kappa\left\langle V, N\right\rangle$, but $\kappa = \|\alpha'\|\ne 0$, so $\left\langle V, N\right\rangle=0$. Therefore $N(s)$ is always perpendicular to $V$. Consider any fixed plane $P$ with $V$ be the normal vector, then $N(s)$ is parallel to $P$.

Conversely, if $N(s)$ parallel to a fixed plane $P$, define $V$ be a normal vector of $P$. This implies $N(s)\perp V$, therefore $\left\langle V, T\right\rangle' = \kappa\left\langle V, N\right\rangle = 0$. So $T$ makes a constant angle with $V$.
\item[(c)]
From the proof above, we have $\left\langle V, N\right\rangle = 0$, so $V = \left\langle V, T\right\rangle T + \left\langle V, B\right\rangle B$. Since $V, \left\langle V, T\right\rangle$ are both constant and the orientation between $T, B$ is fixed, $\left\langle V, B\right\rangle$ is a constant. This implies $B$ makes a constant angle with $V$.

Conversely, let $V$ be the fixed direction, then $0=\left\langle V, B\right\rangle = -\tau\left\langle V, N\right\rangle$. Since $\tau\ne 0$, so $V\perp N$ hence by (b), $T$ makes a constant angle with $V$.
\item[(d)]
\end{enumerate}
\end{proof}

%第六題
\setcounter{theProblemCounter}{5}
\begin{problem}
$\gamma(s)$ 長度參數。若將 $T(s)$ 寫成 $(\sin\phi\cos\theta, \sin\phi\sin\theta, \cos\phi)$, $\phi,\theta$ 是 $s$ 的函數。說明 $\kappa(s)=\sqrt{\phi'^2+\theta'^2\sin^2\phi}$
\end{problem}
\begin{proof}
\begin{align*}
T'(s)&=(\phi'\cos\phi\cos\theta-\theta'\sin\phi\sin\theta,\phi'\cos\phi\sin\theta+\theta'\sin\phi\cos\theta,-\phi'\sin\phi)\\
\rightarrow\kappa(s)&=\left|T'(s)\right|\\
&=\sqrt{\phi'^2\cos^2\phi\cos^2\theta+\theta'^2\sin^2\phi\sin^2\theta+\phi'^2\cos^2\phi\sin^2\theta+\theta'^2\sin^2\phi\cos^2\theta+\phi'^2\sin^2\phi}\\
&=\sqrt{\phi'^2+\theta'^2\sin^2\phi}
\end{align*}
\end{proof}

%第七題
\setcounter{theProblemCounter}{6}
\begin{problem}
$\gamma:\mathbb{R}\to\mathbb{R}^3$,不妨假設是長度參數。
\begin{enumerate}
\item[(b)] 若 $M^tM=I$,$\det(M)=-1$ 且 $\overline{\gamma}=M\gamma$,討論 $\kappa, \tau$ 變化。
\item[(c)] $\overline{\gamma}(s)=\gamma(-s)$,說明 $\kappa, \tau$ 變化。
\end{enumerate}
\end{problem}
\begin{proof}
\begin{enumerate}
\item[(b)]
\begin{align*}
\left|\overline{\gamma}'\right|&=\sqrt{\overline{\gamma}'^T\overline{\gamma}'}\\
&=\sqrt{\gamma'^TM^TM\gamma'}\\
&=\sqrt{\gamma'^T\gamma'}\\
&=\left|\gamma'\right|\\
&=1
\end{align*}
So $s$ is arc-length parameter for $\overline{\gamma}$ too.\\
\begin{align*}
\kappa_{\overline{\gamma}}&=\left|\overline{\gamma}''\right|\\
&=\sqrt{\overline{\gamma}''^T\overline{\gamma}''}\\
&=\sqrt{\gamma''^TM^TM\gamma''}\\
&=\sqrt{\gamma''^T\gamma''}\\
&=\left|\gamma''\right|\\
&=\kappa_{\gamma}
\end{align*}
So $\kappa$ remains the same.
\begin{align*}
N'&=-\kappa T-\tau B\\
\rightarrow \tau&=-\langle N',B\rangle\\
N_{\overline{\gamma}}&=\frac{\overline{\gamma}''}{\kappa_{\overline{\gamma}}}\\
&=M\frac{\gamma''}{\kappa_{\gamma}}\\
&=MN_{\gamma}\\
\rightarrow N'_{\overline{\gamma}}&=MN'_{\gamma}\\
\tau_{\overline{\gamma}}&=-\langle N'_{\overline{\gamma}},B_{\overline{\gamma}}\rangle\\
&=-\langle MN'_{\gamma},T_{\overline{\gamma}}\times N_{\overline{\gamma}}\rangle\\
&=-\langle MN'_{\gamma},\left(MT_{\gamma}\right)\times \left(MN_{\gamma}\right)\rangle\\
&=-\det(M)\langle N'_{\gamma},\left(T_{\gamma}\right)\times \left(N_{\gamma}\right)\rangle\\
&=-\det(M)\langle N'_{\gamma},B_{\gamma}\rangle\\
&=\det(M)\tau_{\gamma}\\
&=-\tau_{\gamma}\\
\end{align*}
So $\tau_{\overline{\gamma}}=-\tau_{\gamma}$.
\item[(c)]
\begin{align*}
\left|\overline{\gamma}'(s)\right|&=\sqrt{\overline{\gamma}'(s)^T\overline{\gamma}'(s)}\\
&=\sqrt{\left(-\gamma'^T(-s)\right)\left(-\gamma'(-s)\right)}\\
&=\sqrt{\gamma'(-s)^T\gamma'(-s)}\\
&=\left|\gamma'(-s)\right|\\
&=1
\end{align*}
So $s$ is arc-length parameter for $\overline{\gamma}$ too.\\
\begin{align*}
\kappa_{\overline{\gamma}}(s)&=\left|\overline{\gamma}''(s)\right|\\
&=\sqrt{\overline{\gamma}''(s)^T\overline{\gamma}''(s)}\\
&=\sqrt{\gamma''(-s)^T\gamma''(-s)}\\
&=\left|\gamma''(-s)\right|\\
&=\kappa_{\gamma}(-s)
\end{align*}
So $\kappa_{\overline{\gamma}}(s)=\kappa_{\gamma}(-s)$.
\begin{align*}
\tau_{\overline{\gamma}}(s)&=\frac{\left|\overline{\gamma}'(s)\ \overline{\gamma}''(s)\ \overline{\gamma}'''(s)\right|}{\left|\overline{\gamma}'(s)\times\overline{\gamma}''(s)\right|^2}\\
&=\frac{\left|-\gamma'(-s)\ \gamma''(-s)\ -\gamma'''(-s)\right|}{\left|-\gamma'(-s)\times\gamma''(-s)\right|^2}\\
&=\frac{\left|\gamma'(-s)\ \gamma''(-s)\ \gamma'''(-s)\right|}{\left|\gamma'(-s)\times\gamma''(-s)\right|^2}\\
&=\tau_{\gamma}(-s)
\end{align*}
So $\tau_{\overline{\gamma}}(s)=\tau_{\gamma}(-s)$.
\end{enumerate}
\end{proof}

%第八題
\setcounter{theProblemCounter}{7}
\begin{problem}
說明 $\overline{\gamma}(u)=\gamma(t(u))$ 時,在對應點%卡恩打錯題目了啦XD
\[
\frac{\det(\overline\gamma',\overline\gamma'',
\overline\gamma''')}{|\overline\gamma'\times \overline\gamma''|^2}(u) = 
\frac{\det(\gamma',\gamma'',\gamma''')}{|\gamma'\times\gamma''|^2}(t)
\]
再用 chain rule 直接說明。
\end{problem}
\begin{proof}
\begin{align*}
\overline{\gamma}'(u)&=\gamma'(t(u))t'(u)\\
\overline{\gamma}''(u)&=\gamma''(t(u))t'(u)^2+\gamma'(t(u))t''(u)\\
\overline{\gamma}'''(u)&=\gamma'''(t(u))t'(u)^3+3\gamma''(t(u))t'(u)t''(u)+\gamma'(t(u))t'''(u)
\end{align*}
\begin{align*}
\rightarrow \det(\overline\gamma',\overline\gamma'',\overline\gamma''')(u)
&=\det(\gamma'(t(u))t'(u),\gamma''(t(u))t'(u)^2+\gamma'(t(u))t''(u),\\ &\hspace{4em}\gamma'''(t(u))t'(u)^3+3\gamma''(t(u))t'(u)t''(u)+\gamma'(t(u))t'''(u))\\
&=\det(\gamma'(t(u))t'(u),\gamma''(t(u))t'(u)^2,\gamma'''(t(u))t'(u)^3+3\gamma''(t(u))t'(u)t''(u))\\
&=\det(\gamma'(t(u))t'(u),\gamma''(t(u))t'(u)^2,\gamma'''(t(u))t'(u)^3)\\
&=t'(u)^6\det(\gamma'(t(u)),\gamma''(t(u)),\gamma'''(t(u)))
\end{align*}
\begin{align*}
|\overline\gamma'\times \overline\gamma''|^2(u) &=|\left(\gamma'(t(u))t'(u)\right)\times\left(\gamma''(t(u))t'(u)^2+\gamma'(t(u))t''(u)\right)|^2\\
&=|\left(\gamma'(t(u))t'(u)\right)\times\left(\gamma''(t(u))t'(u)^2\right)|^2\\
&=t'(u)^6|\gamma'(t(u))\times\gamma''(t(u))|^2
\end{align*}
\begin{align*}
\rightarrow \frac{\det(\overline\gamma',\overline\gamma'',\overline\gamma''')}{|\overline\gamma'\times \overline\gamma''|^2}(u) &= \frac{t'(u)^6\det(\gamma',\gamma'',\gamma''')}{t'(u)^6|\gamma'\times\gamma''|^2}(t)\\
&=\frac{\det(\gamma',\gamma'',\gamma''')}{|\gamma'\times\gamma''|^2}(t)
\end{align*}
\end{proof}
\end{document}
