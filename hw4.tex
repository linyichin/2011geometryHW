\documentclass[10pt,a4paper]{article}
\usepackage{amsmath, amssymb, amsthm}
%加這個就可以設定字體
\usepackage{fontspec}
\usepackage{xkeyval} %MikTeX 2.9 版本相容有誤, 以此修正
%使用xeCJK,其他的還有CJK或是xCJK
\usepackage{xeCJK}
%設定英文字型,不設的話就會使用預設的字型
%\setmainfont{Times New Roman}
%設定中英文的字型
%字型的設定可以使用系統內的字型,而不用像以前一樣另外安裝
%\setCJKmainfont{文泉驛微米黑}
\setCJKmainfont{WenQuanYi Micro Hei}
\setCJKfamilyfont{lh}{LiHei Pro}
\newcommand{\LiHei}{\CJKfamily{lh}}
%中文自動換行
\XeTeXlinebreaklocale "zh"
%文字的彈性間距
\XeTeXlinebreakskip = 0pt plus 1pt
%設定段落之間的距離
\setlength{\parskip}{0.3cm}
%設定行距
%\linespread{1.5}\selectfont

%先暫時用水泥字型,如果任何人看不下去就改吧XD
\usepackage[T1]{fontenc}
\usepackage{concmath}
%\usepackage{mathptmx} %Times

\newcounter{theProblemCounter}
\newtheorem{problem}[theProblemCounter]{Problem}

\begin{document}
\title{{\fontspec{Copperplate Gothic Bold}Geometry Homework 4}}
%\title{{\fontspec{Copperplate}Geometry Homework 4}}
\author{{\it{B96201044}} {\LiHei 黃上恩}, {\it{B98901182}} {\LiHei 時丕勳}, {\it{K0020100x}} {\LiHei 劉士瑋}}
\date{\today}
\maketitle

%第三題
\setcounter{theProblemCounter}{2}
\begin{problem}
\begin{enumerate}
\item[]
\item[(a)] 假設 $\kappa(s)\ne 0, \tau(s)\ne 0$,由四點決定一球,討論空間曲線 $\gamma(s)$ 的密切球,並決定球心與半徑。
\item[(b)] 討論螺線 $(a\cos t, a\sin t, bt)$ 的密切球,$a>0$。
\end{enumerate}
\end{problem}
\begin{proof}
\begin{enumerate}
\item[(a)] Assume that the sphere is $\left|X-C\right| = R$
\begin{align*}
\rightarrow \langle X-C,X-C\rangle &= R^2\\
\rightarrow \langle X-C,X-C\rangle' &= 0\\
&= 2\langle X-C,T\rangle\\
\rightarrow \langle X-C,T\rangle' &= 0\\
&=\langle T,T\rangle+\langle X-C,T'\rangle\\
&=1+\kappa \langle X-C,N\rangle\\
\rightarrow \left(\kappa\langle X-C,N\rangle\right)'&=0\\
&=\kappa'\langle X-C,N\rangle+\kappa\langle T,N\rangle+\kappa\langle X-C,N'\rangle\\
&=\kappa'\langle X-C,N\rangle+\kappa\langle X-C,-\kappa T-\tau B\rangle\\
&=\kappa'\langle X-C,N\rangle-\kappa\tau\langle X-C,B\rangle\\
\rightarrow \langle X-C,T\rangle&=0\\
\langle X-C,N\rangle&=-\frac{1}{\kappa}\\
\langle X-C,B\rangle&=-\frac{\kappa'}{\kappa^2\tau}\\
\rightarrow X-C&=-\frac{1}{\kappa}N-\frac{\kappa'}{\kappa^2\tau}B\\
\rightarrow C&=X+\frac{1}{\kappa}N+\frac{\kappa'}{\kappa^2\tau}B\\
R&=\left|X-C\right|\\
&=\sqrt{\frac{1}{\kappa^2}+\frac{\kappa'^2}{\kappa^4\tau^2}}
\end{align*}
\end{enumerate}
\end{proof}

%第四題
\setcounter{theProblemCounter}{3}
\begin{problem}
$\kappa\ne 0, \tau \ne 0$ 為兩常數,請決定 $\kappa(s)=\kappa, \tau(s)=\tau$ 的曲線方程式。(長度參數 $s$)
\end{problem}
\begin{proof}
Upto translations and rotations, all space curves $\alpha(s)$ satisfying the condition are of the following form
\begin{align*}
\alpha(s) &= (\frac{\kappa}{\kappa^2 + \tau^2}\sin\sqrt{\kappa^2 + \tau^2}s, \frac{\kappa}{\kappa^2 + \tau^2}\cos\sqrt{\kappa^2 + \tau^2}s, \frac{\tau}{\sqrt{\kappa^2 + \tau^2}}s) \\
T(s) &= (\frac{\kappa}{\sqrt{\kappa^2 + \tau^2}}\cos\sqrt{\kappa^2 + \tau^2}s, -\frac{\kappa}{\sqrt{\kappa^2 + \tau^2}}\sin\sqrt{\kappa^2 + \tau^2}s, \frac{\tau}{\sqrt{\kappa^2 + \tau^2}}) \\
\|T(s)\| &= 1 \quad\text{(arc-length)}\\
T'(s) &= (-\kappa\sin\sqrt{\kappa^2 + \tau^2}s), -\kappa\cos\sqrt{\kappa^2 + \tau^2}s, 0) \\
\kappa(s) &= \|T(s)\| = \kappa \\
N(s) &= (-\sin\sqrt{\kappa^2 + \tau^2}s), -\cos\sqrt{\kappa^2 + \tau^2}s, 0) \\
B(s) &= T(s)\times N(s) = (\frac{\tau}{\sqrt{\kappa^2 + \tau^2}}\cos\sqrt{\kappa^2 + \tau^2}s, -\frac{\tau}{\sqrt{\kappa^2 + \tau^2}}\sin\sqrt{\kappa^2 + \tau^2}s, -\frac{\kappa}{\sqrt{\kappa^2 + \tau^2}}) \\
B'(s) &= (-\tau\sin\sqrt{\kappa^2 + \tau^2}s, -\tau\cos\sqrt{\kappa^2 + \tau^2}s, 0) \\
\tau(s) &= B'(s)/N(s) = \tau
\end{align*}
\end{proof}

%第五題
\setcounter{theProblemCounter}{4}
\begin{problem}[Darboux vector]
$\gamma(s)$ arc length
\begin{enumerate}
\item[(a)] 說明 $\exists$ vector $\omega(s)$ (called \emph{Darboux vector}) such that 
\[
\left\{
\begin{array}{ccc}
T' &=& \omega\times T\\
N' &=& \omega\times N\\
B' &=& \omega\times B\end{array}
\right.
\]
\item[(b)] $V(s)$ is a vector along $\gamma(s)$ 且 w.r.t $(T, N,B)$, $V(s)=(v_1(s), v_2(s), v_3(s))$ $\Rightarrow V' = (v_1', v_2', v_3') + \omega \times V$
\item[(c)] 說明 $\omega=\frac12 (T\times T' + N\times N' + B\times B')$
\end{enumerate}
\begin{proof}
\begin{enumerate}
\item[(a)]
Let $\omega(s)=-\tau T+\kappa B$, then:
\begin{align*}
\omega\times T &= \left(-\tau T+\kappa B\right)\times T\\
&=\kappa N\\
&=T'\\
\omega\times N &= \left(-\tau T+\kappa B\right)\times N\\
&=-\kappa T-\tau B\\
&=N'\\
\omega\times B &= \left(-\tau T+\kappa B\right)\times B\\
&=\tau N\\
&=B'
\end{align*}
So $\omega(s)$ satisfy the conditions.
\item[(b)]
\begin{align*}
V&=v_1 T+v_2 N+v_3 B\\
\rightarrow V'&=v_1' T+v_2' N+v_3' B+v_1 T'+v_2 N'+v_3 B'\\
&=v_1' T+v_2' N+v_3' B+v_1\omega\times T+v_2\omega\times N+v_3\omega\times B\\
&=v_1' T+v_2' N+v_3' B+\omega\times\left(v_1T\right)+\omega\times\left(v_2N\right)+\omega\times\left(v_3B\right)\\
&=v_1' T+v_2' N+v_3' B+\omega\times\left(v_1T+v_2N+v_3B\right)\\
&=\left(v_1',v_2',v_3'\right)+\omega\times V
\end{align*}
\item[(c)]
\begin{align*}
\frac12 (T\times T' + N\times N' + B\times B')&=\frac12 (T\times \kappa N + N\times\left(-\kappa T-\tau B\right) + B\times \tau N)\\
&=\frac12 (\kappa B + \kappa B-\tau T - \tau T)\\
&= -\tau T+\kappa B\\
&=\omega
\end{align*}
\end{enumerate}
\end{proof}
\end{problem}

%第八題
\setcounter{theProblemCounter}{7}
\begin{problem}
\begin{enumerate}
\item[]
\item[(a)] 令函數 $x_i:\mathbb{R}^n\to \mathbb{R}. (x_1,\cdots, x_n)\mapsto x_i$。計算 $[dx_i]$,在不同的 $a\in \mathbb{R}^n$, $dx_i$ 如何隨 $a$ 變化。
\item[(b)] 由上題微分式 $df=\frac{\partial f}{\partial x_1} dx_1 + \cdots + \frac{\partial f}{\partial x_n} dx_n$ 與映射 $df$ 結合起來。
\item[(c)] $f:\mathbb{R}^n\to \mathbb{R}^m$,怎麼利用上題幫你計算 $df$
\end{enumerate}
\begin{proof}
\end{proof}
\end{problem}

\end{document}
