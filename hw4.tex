\documentclass[10pt,a4paper]{article}
\usepackage{amsmath, amssymb, amsthm}
%加這個就可以設定字體
\usepackage{fontspec}
\usepackage{xkeyval} %MikTeX 2.9 版本相容有誤, 以此修正
%使用xeCJK,其他的還有CJK或是xCJK
\usepackage{xeCJK}
%設定英文字型,不設的話就會使用預設的字型
%\setmainfont{Times New Roman}
%設定中英文的字型
%字型的設定可以使用系統內的字型,而不用像以前一樣另外安裝
%\setCJKmainfont{文泉驛微米黑}
\setCJKmainfont{WenQuanYi Micro Hei}
\setCJKfamilyfont{lh}{LiHei Pro}
\newcommand{\LiHei}{\CJKfamily{lh}}
%中文自動換行
\XeTeXlinebreaklocale "zh"
%文字的彈性間距
\XeTeXlinebreakskip = 0pt plus 1pt
%設定段落之間的距離
\setlength{\parskip}{0.3cm}
%設定行距
%\linespread{1.5}\selectfont

%先暫時用水泥字型,如果任何人看不下去就改吧XD
\usepackage[T1]{fontenc}
\usepackage{concmath}
%\usepackage{mathptmx} %Times

\newcounter{theProblemCounter}
\newtheorem{problem}[theProblemCounter]{Problem}

\begin{document}
%\title{{\fontspec{Copperplate Gothic Bold}Geometry Homework 4}}
\title{{\fontspec{Copperplate}Geometry Homework 4}}
\author{{\it{B96201044}} {\LiHei 黃上恩}, {\it{B98901182}} {\LiHei 時丕勳}, {\it{K0020100x}} {\LiHei 劉士瑋}}
\date{\today}
\maketitle

%第三題
\setcounter{theProblemCounter}{2}
\begin{problem}
\begin{enumerate}
\item[]
\item[(a)] 假設 $\kappa(s)\ne 0, \tau(s)\ne 0$,由四點決定一球,討論空間曲線 $\gamma(s)$ 的密切球,並決定球心與半徑。
\item[(b)] 討論螺線 $(a\cos t, a\sin t, bt)$ 的密切球,$a>0$。
\end{enumerate}
\end{problem}
\begin{proof}
\end{proof}

%第四題
\setcounter{theProblemCounter}{3}
\begin{problem}
$\kappa\ne 0, \tau \ne 0$ 為兩常數,請決定 $\kappa(s)=\kappa, \tau(s)=\tau$ 的曲線方程式。(長度參數 $s$)
\end{problem}
\begin{proof}
Upto translations and rotations, all space curves $\alpha(s)$ satisfying the condition are of the following form
\begin{align*}
\alpha(s) &= (\frac{\kappa}{\kappa^2 + \tau^2}\sin\sqrt{\kappa^2 + \tau^2}s, \frac{\kappa}{\kappa^2 + \tau^2}\cos\sqrt{\kappa^2 + \tau^2}s, \frac{\tau}{\sqrt{\kappa^2 + \tau^2}}s) \\
T(s) &= (\frac{\kappa}{\sqrt{\kappa^2 + \tau^2}}\cos\sqrt{\kappa^2 + \tau^2}s, -\frac{\kappa}{\sqrt{\kappa^2 + \tau^2}}\sin\sqrt{\kappa^2 + \tau^2}s, \frac{\tau}{\sqrt{\kappa^2 + \tau^2}}) \\
\|T(s)\| &= 1 \quad\text{(arc-length)}\\
T'(s) &= (-\kappa\sin\sqrt{\kappa^2 + \tau^2}s), -\kappa\cos\sqrt{\kappa^2 + \tau^2}s, 0) \\
\kappa(s) &= \|T(s)\| = \kappa \\
N(s) &= (-\sin\sqrt{\kappa^2 + \tau^2}s), -\cos\sqrt{\kappa^2 + \tau^2}s, 0) \\
B(s) &= T(s)\times N(s) = (\frac{\tau}{\sqrt{\kappa^2 + \tau^2}}\cos\sqrt{\kappa^2 + \tau^2}s, -\frac{\tau}{\sqrt{\kappa^2 + \tau^2}}\sin\sqrt{\kappa^2 + \tau^2}s, -\frac{\kappa}{\sqrt{\kappa^2 + \tau^2}}) \\
B'(s) &= (-\tau\sin\sqrt{\kappa^2 + \tau^2}s, -\tau\cos\sqrt{\kappa^2 + \tau^2}s, 0) \\
\tau(s) &= B'(s)/N(s) = \tau
\end{align*}
\end{proof}

%第五題
\setcounter{theProblemCounter}{4}
\begin{problem}[Darboux vector]
$\gamma(s)$ arc length
\begin{enumerate}
\item[(a)] 說明 $\exists$ vector $\omega(s)$ (called \emph{Darboux vector}) such that 
\[
\left\{
\begin{array}{ccc}
T' &=& \omega\times T\\
N' &=& \omega\times N\\
B' &=& \omega\times B\end{array}
\right.
\]
\item[(b)] $V(s)$ is a vector along $\gamma(s)$ 且 w.r.t $(T, N,B)$, $V(s)=(v_1(s), v_2(s), v_3(s))$ $\Rightarrow V' = (v_1', v_2', v_3') + \omega \times V$
\item[(c)] 說明 $\omega=\frac12 (T\times T' + N\times N' + B\times B')$
\end{enumerate}
\end{problem}

%第八題
\setcounter{theProblemCounter}{7}
\begin{problem}
\begin{enumerate}
\item[]
\item[(a)] 令函數 $x_i:\mathbb{R}^n\to \mathbb{R}. (x_1,\cdots, x_n)\mapsto x_i$。計算 $[dx_i]$,在不同的 $a\in \mathbb{R}^n$, $dx_i$ 如何隨 $a$ 變化。
\item[(b)] 由上題微分式 $df=\frac{\partial f}{\partial x_1} dx_1 + \cdots + \frac{\partial f}{\partial x_n} dx_n$ 與映射 $df$ 結合起來。
\item[(c)] $f:\mathbb{R}^n\to \mathbb{R}^m$,怎麼利用上題幫你計算 $df$
\end{enumerate}
\end{problem}

\end{document}
