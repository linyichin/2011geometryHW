\documentclass[10pt,a4paper]{article}
\usepackage{amsmath, amssymb, amsthm}
%加這個就可以設定字體
\usepackage{fontspec}
\usepackage{xkeyval} %MikTeX 2.9 版本相容有誤, 以此修正
%使用xeCJK,其他的還有CJK或是xCJK
\usepackage{xeCJK}
%設定英文字型,不設的話就會使用預設的字型
%\setmainfont{Times New Roman}
%設定中英文的字型
%字型的設定可以使用系統內的字型,而不用像以前一樣另外安裝
%\setCJKmainfont{文泉驛微米黑}
\setCJKmainfont{WenQuanYi Micro Hei}
\setCJKfamilyfont{lh}{LiHei Pro}
\newcommand{\LiHei}{\CJKfamily{lh}}
%中文自動換行
\XeTeXlinebreaklocale "zh"
%文字的彈性間距
\XeTeXlinebreakskip = 0pt plus 1pt
%設定段落之間的距離
\setlength{\parskip}{0.3cm}
%設定行距
%\linespread{1.5}\selectfont
\usepackage{enumerate}

%先暫時用水泥字型,如果任何人看不下去就改吧XD
\usepackage[T1]{fontenc}
\usepackage{concmath}
%\usepackage{mathptmx} %Times

\newcounter{theProblemCounter}
\newtheorem{problem}[theProblemCounter]{Problem}

\begin{document}
\title{{\fontspec{Copperplate Gothic Bold}Geometry Homework 9}}
%\title{{\fontspec{Copperplate}Geometry Homework 9}}
\author{{\it{B96201044}} {\LiHei 黃上恩}, {\it{B98901182}} {\LiHei 時丕勳}, {\it{K0020100x}} {\LiHei 劉士瑋}}
\date{\today}
\maketitle

\newcommand{\bx}{\mathbb{X}}
\newcommand{\bfx}{\mathbf{X}}
\newcommand{\grad}{\textrm{grad }}
\newcommand{\sech}{\mbox{sech}}
%\newcommand{\cosh}{\mbox{cosh}\ }
%\newcommand{\tanh}{\mbox{tanh}\ }
%\newcommand{\sinh}{\mbox{sinh}\ }
%第四題
\setcounter{theProblemCounter}{3}
\begin{problem}[Ex p.101 14]
\begin{enumerate}
(Gradient on Surfaces.) The gradient of a differentiable function $f: S\mapsto \mathbb{R}$ is a differentiable map $\textrm{grad }f: S\mapsto\mathbb{R}^3$ which assigns to each point $p\in S$ a vector $\textrm{grad }f(p)\in T_p(S)\subset \mathbb{R}^3$ such that
\[
\langle \textrm{grad }f(p), v\rangle_p=df_p(v)\hspace{2em}\textrm{ for all }v\in T_p(S)    
\]
Show that
\item[(a)]
If $E, F, G$ are the coefficients of the first fundamental form in a parametrization $\bfx: U\subset\mathbb{R}^2\mapsto S$, then $\textrm{grad }f$ on $\bfx(U)$ is given by
\[
\textrm{grad }f=\frac{f_uG-f_vF}{EG-F^2}\bfx_u+\frac{f_vE-f_uF}{EG-F^2}\bfx_v
\]
In particular, if $S=\mathbb{R}^2$ with coordinates $x, y$,
\[
\textrm{grad }f=f_xe_1+f_ye_2
\]
where $\{e_1, e_2\}$ is the canonical basis of $\mathbb{R}^2$ (thus, the definition agrees with the usual definition of gradient in the plane)
\item[(b)]
為什麼不直接將 $\textrm{gradient }f$ 定義成 $f_u\bx_u+f_v\bx_v$,這有什麼缺點(例如座標變換)
\end{enumerate}
\end{problem}
\begin{proof}
\begin{enumerate}
\item[(a)]
First, 
\begin{align*}
\langle \grad f(p), \bfx_u\rangle_p = df_p(\bfx_u) = f_u\\
\langle \grad f(p), \bfx_v\rangle_p = df_p(\bfx_v) = f_v
\end{align*}
Let $\grad f = q\bfx_u + r\bfx_v$. Then
\begin{align*}
\langle \grad f(p), \bfx_u\rangle = Eq + Fr = f_u\\
\langle \grad f(p), \bfx_v\rangle = Fq + Gr = f_v
\end{align*}
Therefore, solve the linear equations and get
\begin{align*}
q = \frac{f_uG - f_uF}{EG - F^2};\\
r = \frac{f_vE - f_uF}{EG - F^2}
\end{align*}
Then the two results follow immediately.
\item[(b)]
If we define the gradient in that way, let $S = \mathbb{R}^2$ be the surface and $\mathbf{X}(u, v) = (u, v)$, $\mathbf{Y}(s, t) = (s, s+t)$ be its two parametrizations. If $f(u, v) = v$, then $f(s, t) = s+t$ and therefore $\grad f = \bfx_v = \mathbf{Y}_s + \mathbf{Y}_t$. But clearly $\bfx_v = (0, 1) \neq (1, 2) = \mathbf{Y}_s + \mathbf{Y}_t$, which is a contradiction.
\end{enumerate}
\end{proof}

%第七題
\setcounter{theProblemCounter}{6}
\begin{problem}
計算下列 surface 的 $\Gamma_{ij}^k$ (共有六項)
\begin{enumerate}
\item[(b)]
$(x(t),y(t)\cos\theta,y(t)\sin\theta)$
\item[(c)]
$E=G=\lambda^2, F=0$
\end{enumerate}
\end{problem}
\begin{proof}
\begin{enumerate}
\item[(b)]
Let $u=t, v=\theta$.
\begin{align*}
&\bx_u=(x_u,y_u\cos v,y_u\sin v), \bx_v=(0,-y\sin v,y\cos v)\\
&\rightarrow E=x_u^2+y_u^2, F=0, G=y^2\\
&{[}1,1,1]=\frac{E_u}{2}, {[}1,1,2]=-\frac{E_v}{2}, {[}1,2,1]=\frac{E_v}{2}\\
&{[}1,2,2]=\frac{G_u}{2}, {[}2,2,1]=-\frac{G_u}{2}, {[}2,2,2]=\frac{G_v}{2}\\
&\Gamma_{11}^1=\frac{E_u}{2E}=\frac{x_ux_{uu}+y_uy_{uu}}{x_u^2+y_u^2}\\
&\Gamma_{11}^2=-\frac{E_v}{2G}=0\\
&\Gamma_{12}^1=\frac{E_v}{2E}=0\\
&\Gamma_{12}^2=\frac{G_u}{2G}=\frac{y_u}{y}\\
&\Gamma_{22}^1=-\frac{G_u}{2E}=\frac{yy_u}{x_u^2+y_u^2}\\
&\Gamma_{22}^2=\frac{G_v}{2G}=0
\end{align*}
\item[(c)]
\begin{align*}
&g^{11}=\frac{1}{\lambda^2}, g^{22}=\frac{1}{\lambda^2}, g^{12}=g^{21}=0\\
&[1,1,1]=\frac{E_u}{2}, [1,1,2]=-\frac{E_v}{2}, [1,2,1]=\frac{E_v}{2}\\
&[1,2,2]=\frac{G_u}{2}, [2,2,1]=-\frac{G_u}{2}, [2,2,2]=\frac{G_v}{2}\\
&\Gamma_{11}^1=\frac{E_u}{2E}=\frac{\lambda_u}{\lambda}\\
&\Gamma_{11}^2=-\frac{E_v}{2G}=\frac{\lambda_v}{\lambda}\\
&\Gamma_{12}^1=\frac{E_v}{2E}=\frac{\lambda_v}{\lambda}\\
&\Gamma_{12}^2=\frac{G_u}{2G}=\frac{\lambda_u}{\lambda}\\
&\Gamma_{22}^1=-\frac{G_u}{2E}=\frac{\lambda_u}{\lambda}\\
&\Gamma_{22}^2=\frac{G_v}{2G}=\frac{\lambda_v}{\lambda}
\end{align*}
\end{enumerate}
\end{proof}

%第八題
\setcounter{theProblemCounter}{7}
\begin{problem}[Ex p.237 1, 2]
\begin{enumerate}
\item[(a)]
Show that if $\bfx$ is an orthogonal parametrization, that is, $F=0$, then
\[
K=-\frac{1}{2\sqrt{EG}}\left\{\left(\frac{E_v}{\sqrt{EG}}\right)_v+\left(\frac{G_u}{\sqrt{EG}}\right)_u\right\}
\]
\item[(b)]
Show that if $\bfx$ is an isothermal parametrization, that is, $E=G=\lambda(u,v)$ and $F=0$, then
\[
K=-\frac{1}{2\lambda}\Delta(\log\lambda)
\]
where $\Delta\phi$ denotes the Laplacian $(\partial^2\phi/\partial u^2)+(\partial^2\phi/\partial v^2)$ of the function $\phi$. Conclude that when $E=G=(u^2+v^2+c)^{-2}$ and $F=0$, then $K=\textrm{const}.=4c$.
\end{enumerate}
\end{problem}
\begin{proof}
\begin{enumerate}
\item[(a)]
\begin{align*}
&g^{11}=\frac{1}{E}, g^{22}=\frac{1}{G}, g^{12}=g^{21}=0\\
&[1,1,1]=\frac{E_u}{2}, [1,1,2]=-\frac{E_v}{2}, [1,2,1]=\frac{E_v}{2}\\
&[1,2,2]=\frac{G_u}{2}, [2,2,1]=-\frac{G_u}{2}, [2,2,2]=\frac{G_v}{2}\\
&\Gamma_{11}^1=\frac{E_u}{2E}, \Gamma_{11}^2=-\frac{E_v}{2G}\\
&\Gamma_{12}^1=\frac{E_v}{2E}, \Gamma_{12}^2=\frac{G_u}{2G}\\
&\Gamma_{22}^1=-\frac{G_u}{2E}, \Gamma_{22}^2=\frac{G_v}{2G}
\end{align*}
\begin{align*}
R_{112}^2&=\Gamma_{11,2}^2-\Gamma_{12,1}^2+\Gamma_{11}^1\Gamma_{21}^2+\Gamma_{11}^2\Gamma_{22}^2-\Gamma_{12}^1\Gamma_{11}^2-\Gamma_{12}^2\Gamma_{12}^2\\
&=-\left(\frac{E_v}{2G}\right)_v-\left(\frac{G_u}{2G}\right)_u+
\frac{E_u}{2E}\frac{G_u}{2G}-\frac{E_v}{2G}\frac{G_v}{2G}+\frac{E_v}{2E}\frac{E_v}{2G}-\frac{G_u}{2G}\frac{G_u}{2G}\\
&=-\frac{GE_{vv}-G_vE_v}{2G^2}-\frac{GG_{uu}-G_uG_u}{2G^2}+
\frac{E_u}{2E}\frac{G_u}{2G}-\frac{E_v}{2G}\frac{G_v}{2G}+\frac{E_v}{2E}\frac{E_v}{2G}-\frac{G_u}{2G}\frac{G_u}{2G}\\
&=\frac{1}{4G^2}\left(-2GE_{vv}+G_vE_v-2GG_{uu}+G_uG_u+\frac{GE_u}{E}G_u+\frac{GE_v}{E}E_v\right)\\
K&=\frac{R_{1212}}{EG}\\
&=\frac{R_{112}^2}{E}\\
&=\frac{1}{4EG^2}\left(-2GE_{vv}+G_vE_v-2GG_{uu}+G_uG_u+\frac{GE_u}{E}G_u+\frac{GE_v}{E}E_v\right)\\
&=\frac{1}{4}\left(-2\frac{E_{vv}}{EG}-2\frac{G_{uu}}{EG}+\frac{E_vG_v}{EG^2}+\frac{E_uG_u}{E^2G}+\frac{G_uG_u}{EG^2}+\frac{E_vE_v}{E^2G}\right)
\end{align*}
\begin{align*}
&-\frac{1}{2\sqrt{EG}}\left\{\left(\frac{E_v}{\sqrt{EG}}\right)_v+\left(\frac{G_u}{\sqrt{EG}}\right)_u\right\}\\
&=-\frac{1}{2\sqrt{EG}}\left\{\frac{\sqrt{EG}E_{vv}-\frac{E_vG+EG_v}{2\sqrt{EG}}E_v}{EG}+\frac{\sqrt{EG}G_{uu}-\frac{E_uG+EG_u}{2\sqrt{EG}}G_u}{EG}\right\}\\
&=-\frac{1}{2}\left\{\frac{E_{vv}-\frac{E_vG+EG_v}{2EG}E_v}{EG}+\frac{G_{uu}-\frac{E_uG+EG_u}{2EG}G_u}{EG}\right\}\\
&=\frac{1}{4}\left(-2\frac{E_{vv}}{EG}-2\frac{G_{uu}}{EG}+\frac{E_vG_v}{EG^2}+\frac{E_uG_u}{E^2G}+\frac{G_uG_u}{EG^2}+\frac{E_vE_v}{E^2G}\right)\\
\end{align*}
So $K=-\frac{1}{2\sqrt{EG}}\left\{\left(\frac{E_v}{\sqrt{EG}}\right)_v+\left(\frac{G_u}{\sqrt{EG}}\right)_u\right\}$.
\item[(b)]
\begin{align*}
K&=-\frac{1}{2\sqrt{EG}}\left\{\left(\frac{E_v}{\sqrt{EG}}\right)_v+\left(\frac{G_u}{\sqrt{EG}}\right)_u\right\}\\
&=-\frac{1}{2\lambda}\left\{\left(\frac{\lambda_v}{\lambda}\right)_v+\left(\frac{\lambda_u}{\lambda}\right)_u\right\}\\
\Delta(\log\lambda)&=(\log\lambda)_{uu}+(\log\lambda)_{vv}\\
&=\left(\frac{\lambda_u}{\lambda}\right)_u+\left(\frac{\lambda_v}{\lambda}\right)_v\\
\rightarrow K&=-\frac{1}{2\lambda}\Delta(\log\lambda)
\end{align*}
\end{enumerate}
\end{proof}
\end{document}
