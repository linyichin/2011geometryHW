\documentclass[10pt,a4paper]{article}
\usepackage{amsmath, amssymb, amsthm}
%加這個就可以設定字體
\usepackage{fontspec}
\usepackage{xkeyval} %MikTeX 2.9 版本相容有誤, 以此修正
%使用xeCJK,其他的還有CJK或是xCJK
\usepackage{xeCJK}
%設定英文字型,不設的話就會使用預設的字型
%\setmainfont{Times New Roman}
%設定中英文的字型
%字型的設定可以使用系統內的字型,而不用像以前一樣另外安裝
\setCJKmainfont{文泉驛微米黑}
\setCJKfamilyfont{lh}{LiHei Pro}
\newcommand{\LiHei}{\CJKfamily{lh}}
%中文自動換行
\XeTeXlinebreaklocale "zh"
%文字的彈性間距
\XeTeXlinebreakskip = 0pt plus 1pt
%設定段落之間的距離
\setlength{\parskip}{0.3cm}
%設定行距
%\linespread{1.5}\selectfont

%先暫時用水泥字型,如果任何人看不下去就改吧XD
\usepackage[T1]{fontenc}
\usepackage{concmath}
%\usepackage{mathptmx} %Times

\newcounter{theProblemCounter}
\newtheorem{problem}[theProblemCounter]{Problem}


\begin{document}
\title{{\fontspec{Copperplate Gothic Bold}Geometry Homework 2}}
\author{{\it{B96201044}} {\LiHei 黃上恩}, {\it{B98901182}} {\LiHei 時丕勳}, {\it{K0020100x}} {\LiHei 劉士瑋}}
\date{\today}
\maketitle

%第三題
\setcounter{theProblemCounter}{2}
\begin{problem}[P47: 5]
If a closed plane curve $C$ is contained inside a disk of radius $r$, prove that there exists a point $p\in C$ such that the curvature $\kappa$ of $C$ at $p$ satifies $|\kappa|\ge 1/r$.
\end{problem}
\begin{proof}
Let $X(s)$ denote the curve $C$, where $s\in [0, l]$ is an arc-length parameter, that is, $\|X'(s)\|\equiv 1$. Since $C$ is contained inside a disk of radius $r$, let $A$ be the centre of the disk. So we have
\begin{equation} \|X(s)-A\|\le r\end{equation}
Consider $f(s)=\left\langle X(s)-A, X(s)-A\right\rangle$. Since $[0, l]$ is compact, the maximum exists, denoting by $f(s') = \max_{s\in[0,l]}f(s)$. Therefore, we have $f'(s')=0$ and $f''(s') \le 0$. Now 
\begin{equation}
	f''(s)=2\left(\|X'(s)\|^2+\kappa(s)\left\langle X(s)-A, N(s)\right\rangle\right),
\end{equation}
where $X''(s)=\kappa(s)N(s)$ and $N(s)$ is the normal vector. Take $s=s'$ in (2) we have $f''(s') \le 0$ and hence
\begin{equation}
\kappa(s')\left\langle X(s)-A, N(s)\right\rangle \le -1
\end{equation}
This implies
\begin{equation}
|\kappa(s')\left\langle X(s)-A, N(s)\right\rangle| \ge 1
\end{equation}
By (1), $|\left\langle X(s)-A, N(s)\right\rangle| \le \|X(s)-A\|\cdot \|N(s)\| \le r$. We have $|\kappa(s')|\ge 1/r$ as desired.
\end{proof}

%第四題 %我又打錯題目了= =
\setcounter{theProblemCounter}{3}
\begin{problem}[P23: 4, 僅討論平面情形]
Assume that all normals of a parame-trized curve pass through a fixed point. Prove that the trace of the curve is contained in a circle.
\end{problem}

%第五題
\setcounter{theProblemCounter}{4}
\begin{problem}
以 $t=0$ 開始將曲線 $(t^2, t^3)$ 化成長度參數。並討論 $t=0$ 時的曲率。
\end{problem}

%第六題
\setcounter{theProblemCounter}{5}
\begin{problem}
\begin{enumerate}
\item[]
\item[(a)] 以原點為中心,將 $y=f(x)$ 的圖形縮放 $\lambda$ 倍,並說明新的圖形是 $y=\lambda f(\frac{x}{\lambda})$ 的函數圖形。
\item[(b)] 討論曲率的變化。
\end{enumerate}
\end{problem}

%第七題
\setcounter{theProblemCounter}{6}
\begin{problem}
如圖,有一橢圓,其焦點為 $O_1$ 和 $O_2$,設 $L$ 切橢圓於 $P$,且 $L$ 與 $\overline{O_2P}$ 之夾角為 $\theta$。以 $\theta$ 為參數,說明曲率 $\kappa\propto\sin^3\theta$
\end{problem}

%第九題
\setcounter{theProblemCounter}{8}
\begin{problem}
如圖,有 regular curve $\gamma(t)$,$\gamma_0=\gamma(0)$,$N_0=N(0)$,$L_0=\{\gamma_0+vN_0\}$。現考慮直線 $L_t=\{\gamma(t)+uN(t)\}$,令 $P(t)=L_t\cap L_0$ 證明
\[\kappa(0)\ne 0\Rightarrow \lim_{t\to 0}P(t)=\gamma_0 + \frac{1}{\kappa(0)}N_0\]
\end{problem}
\end{document}
