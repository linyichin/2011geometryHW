\documentclass[10pt,a4paper]{article}
\usepackage{amsmath, amssymb, amsthm}
%加這個就可以設定字體
\usepackage{fontspec}
\usepackage{xkeyval} %MikTeX 2.9 版本相容有誤, 以此修正
%使用xeCJK,其他的還有CJK或是xCJK
\usepackage{xeCJK}
%設定英文字型,不設的話就會使用預設的字型
%\setmainfont{Times New Roman}
%設定中英文的字型
%字型的設定可以使用系統內的字型,而不用像以前一樣另外安裝
\setCJKmainfont{LiHei Pro}
%中文自動換行
\XeTeXlinebreaklocale "zh"
%文字的彈性間距
\XeTeXlinebreakskip = 0pt plus 1pt
%設定段落之間的距離
\setlength{\parskip}{0.3cm}
%設定行距
%\linespread{1.5}\selectfont

%先暫時用水泥字型,如果任何人看不下去就改吧XD
\usepackage[T1]{fontenc}
\usepackage{concmath}
%\usepackage{mathptmx} %Times

\newcounter{theProblemCounter}
\newtheorem{problem}[theProblemCounter]{Problem}


\begin{document}
\title{{\fontspec{Copperplate Gothic Bold}Geometry Homework 2}}
\author{{\it{B96201044}} 黃上恩, {\it{B98901182}} 時丕勳, {\it{K0020100x}} 劉士瑋}
\date{\today}
\maketitle

%第三題
\setcounter{theProblemCounter}{2}
\begin{problem}[P47: 5]
If a closed plane curve $C$ is contained inside a disk of radius $r$, prove that there exists a point $p\in C$ such that the curvature $\kappa$ of $C$ at $p$ satifies $|\kappa|\ge 1/r$.
\end{problem}
\begin{proof}
\end{proof}

%第四題
\setcounter{theProblemCounter}{3}
\begin{problem}[P23: 4, 僅討論平面情形]
Assume that all parametrized curve $\alpha$ has the property that all its tangent lines pass through a fixed point.
\begin{enumerate}
\item[(a)] Prove that the trace of $\alpha$ is a (segment of a) straight line.
\item[(b)] Does the conclusion in part (a) still hold if $\alpha$ is not regular?
\end{enumerate}
\end{problem}

%第五題
\setcounter{theProblemCounter}{4}
\begin{problem}
以 $t=0$ 開始將曲線 $(t^2, t^3)$ 化成長度參數。並討論 $t=0$ 時的曲綠。
\end{problem}

%第六題
\setcounter{theProblemCounter}{5}
\begin{problem}
\begin{enumerate}
\item[]
\item[(a)] 以原點為中心,將 $y=f(x)$ 的圖形縮放 $\lambda$ 倍,並說明新的圖形是 $y=\lambda f(\frac{x}{\lambda})$ 的函數圖形。
\item[(b)] 討論曲率的變化。
\end{enumerate}
\end{problem}

%第七題
\setcounter{theProblemCounter}{6}
\begin{problem}
如圖,有一橢圓,其焦點為 $O_1$ 和 $O_2$,設 $L$ 切橢圓於 $P$,且 $L$ 與 $\overline{O_2P}$ 之夾角為 $\theta$。以 $\theta$ 為參數,說明曲率 $\kappa\propto\sin^3\theta$
\end{problem}

%第九題
\setcounter{theProblemCounter}{8}
\begin{problem}
如圖,有 regular curve $\gamma(t)$,$\gamma_0=\gamma(0)$,$N_0=N(0)$,$L_0=\{\gamma_0+vN_0\}$。現考慮直線 $L_t=\{\gamma(t)+uN(t)\}$,令 $P(t)=L_t\cap L_0$ 證明
\[\kappa(0)\ne 0\Rightarrow \lim_{t\to 0}P(t)=\gamma_0 + \frac{1}{\kappa(0)}N_0\]
\end{problem}
\end{document}
