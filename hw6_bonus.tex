\documentclass[10pt,a4paper]{article}
\usepackage{amsmath, amssymb, amsthm}
%加這個就可以設定字體
\usepackage{fontspec}
\usepackage{xkeyval} %MikTeX 2.9 版本相容有誤, 以此修正
%使用xeCJK,其他的還有CJK或是xCJK
\usepackage{xeCJK}
%設定英文字型,不設的話就會使用預設的字型
%\setmainfont{Times New Roman}
%設定中英文的字型
%字型的設定可以使用系統內的字型,而不用像以前一樣另外安裝
%\setCJKmainfont{文泉驛微米黑}
\setCJKmainfont{WenQuanYi Micro Hei}
\setCJKfamilyfont{lh}{LiHei Pro}
\newcommand{\LiHei}{\CJKfamily{lh}}
%中文自動換行
\XeTeXlinebreaklocale "zh"
%文字的彈性間距
\XeTeXlinebreakskip = 0pt plus 1pt
%設定段落之間的距離
\setlength{\parskip}{0.3cm}
%設定行距
%\linespread{1.5}\selectfont
\usepackage{enumerate}

%先暫時用水泥字型,如果任何人看不下去就改吧XD
\usepackage[T1]{fontenc}
\usepackage{concmath}
%\usepackage{mathptmx} %Times

\usepackage{tabularx}

\newcounter{theProblemCounter}
\newtheorem{problem}[theProblemCounter]{Problem}

\begin{document}
<<<<<<< HEAD
\title{{\fontspec{Copperplate Gothic Bold}Geometry Homework Bonus}}
%\title{{\fontspec{Copperplate}Geometry Homework Bonus}}
=======
%\title{{\fontspec{Copperplate Gothic Bold}Geometry Homework Bonus}}
\title{{\fontspec{Copperplate}Geometry Homework Bonus}}
>>>>>>> a7c00a1ad86aa98b651be2ba77dc50cf3d8e46d1
\author{{\it{B96201044}} {\LiHei 黃上恩}, {\it{B98901182}} {\LiHei 時丕勳}, {\it{K0020100x}} {\LiHei 劉士瑋}}
\date{\today}
\maketitle

\newcommand{\bx}{\mathbb{X}}
\newcommand{\bfx}{\mathbf{x}}
\newcommand{\grad}{\textrm{grad }}
\newcommand{\sech}{\mbox{sech}}
\newcommand{\pr}[2]{\frac{\partial #1}{\partial #2}}
\newcommand{\prr}[3]{\frac{\partial^2 #1}{\partial #2\partial #3}}
\newcommand{\ip}[2]{\langle#1, #2\rangle}

<<<<<<< HEAD
%第?題
%\setcounter{theProblemCounter}{3}
=======
%第四題
\setcounter{theProblemCounter}{3}
>>>>>>> a7c00a1ad86aa98b651be2ba77dc50cf3d8e46d1
\begin{problem}
給定一 simple close curve $\gamma(s)$,考慮$\gamma(s)$沿著$N$方向以$|\kappa|$的變化率變動的情形。 Formulation: 令 $F(s, t) = (x(s, t), y(s, t))$\\ $F(s,t)$ 可想成 $\gamma_t(s)$,是在時間$t$時的曲線。其中$\gamma_0(s) = \gamma(s)$,此時不妨假設$s$為長度參數。但當$t \neq 0$, $s$並非$\gamma_t(s)$之長度參數。\begin{enumerate}
\item $\gamma_0(s) = (\cos s, \sin s)$ 時,說明$\gamma_t(s)$的變化
\item 令$\Delta=\sqrt{\left(\frac{dx}{ds}\right)^2 + \left(\frac{dy}{ds}\right)^2}$, 證明$\frac{\partial\Delta}{\partial t}=-\kappa_t^2(s)\Delta$ \par
(其實$\frac{dx}{ds}, \frac{dy}{ds}$是$\frac{\partial x}{\partial s}, \frac{\partial y}{\partial s}$)
\item 說明 $\gamma_t$的長度會越來越短。
\item 給出一個想法,說明若$\gamma_1, \gamma_2$一開始不相交,則$\gamma_{1t}, \gamma_{2t}$就永遠不會相交。
\end{enumerate}
\end{problem}
\begin{proof}
\begin{enumerate}
\item 
<<<<<<< HEAD
\begin{align*}
F &= (\sin s, \cos s)\\
F_s &= \Delta T \\
F_t &= \kappa N \\
\end{align*}
=======
$$
F = (\sin s, \cos s)
F_s &= \Delta T \\
F_t &= \kappa N \\
$$
>>>>>>> a7c00a1ad86aa98b651be2ba77dc50cf3d8e46d1
\item 
\begin{align*}
F_s &= \Delta T \\
F_t &= \kappa N \\
F_{ss} &= \Delta^2\kappa N + \Delta_sT\\
\pr{\Delta}{t}&=\frac{\ip{F_s}{F_s}_t}{2\Delta}=\frac{\ip{F_{st}}{F_s}}{\Delta}=\frac{\ip{F_t}{F_s}_s-\ip{F_{t}}{F_{ss}}}{\Delta}=-\frac{\ip{F_t}{F_{ss}}}{\Delta}=-\kappa^2\Delta
\end{align*}
\item Denote the length of $\gamma_t$ by $|\gamma_t|$ and let $|\gamma_0|=\lambda$. Then by 2., $$\pr{|\gamma_t|}{t} =\int_0^{\lambda}\pr{\Delta}{t}\Delta ds = \int_0^{\lambda}-\kappa_t^2(s)\Delta^2 ds \leq 0.$$
\end{enumerate}
\end{proof}
\end{document}
