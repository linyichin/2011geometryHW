\documentclass[10pt,a4paper]{article}
\usepackage{amsmath, amssymb, amsthm}
%加這個就可以設定字體
\usepackage{fontspec}
\usepackage{xkeyval} %MikTeX 2.9 版本相容有誤, 以此修正
%使用xeCJK,其他的還有CJK或是xCJK
\usepackage{xeCJK}
%設定英文字型,不設的話就會使用預設的字型
%\setmainfont{Times New Roman}
%設定中英文的字型
%字型的設定可以使用系統內的字型,而不用像以前一樣另外安裝
%\setCJKmainfont{文泉驛微米黑}
\setCJKmainfont{WenQuanYi Micro Hei}
\setCJKfamilyfont{lh}{LiHei Pro}
\newcommand{\LiHei}{\CJKfamily{lh}}
%中文自動換行
\XeTeXlinebreaklocale "zh"
%文字的彈性間距
\XeTeXlinebreakskip = 0pt plus 1pt
%設定段落之間的距離
\setlength{\parskip}{0.3cm}
%設定行距
%\linespread{1.5}\selectfont

%先暫時用水泥字型,如果任何人看不下去就改吧XD
\usepackage[T1]{fontenc}
\usepackage{concmath}
%\usepackage{mathptmx} %Times

\newcounter{theProblemCounter}
\newtheorem{problem}[theProblemCounter]{Problem}

\begin{document}
%\title{{\fontspec{Copperplate Gothic Bold}Geometry Homework 5}}
\title{{\fontspec{Copperplate}Geometry Homework 5}}
\author{{\it{B96201044}} {\LiHei 黃上恩}, {\it{B98901182}} {\LiHei 時丕勳}, {\it{K0020100x}} {\LiHei 劉士瑋}}
\date{\today}
\maketitle

\newcommand{\bx}{\mathbb{X}}
%第一題
\setcounter{theProblemCounter}{0}
\begin{problem}[Ex P151 2]
Show that if a surface is tangent to a plane along a curve, then the points of this curve are either parabolic or planar.
\end{problem}
\begin{proof}
Assume that the curve is $\gamma(s)$, then along this curve, $N(\gamma(s))$ is perpendicular to the plane, so it is constant.\\
At point $\gamma(s)$, $[dN](\gamma'(s))=\left(\frac{d N(\gamma(t))}{dt}\right)_{t=s}=0$, so $\gamma'(s)$ is one of the principal direction of the surface at $\gamma(s)$, and it's associated principal curvature is $0$.\\
So the gaussian curvature of the surface at $\gamma(s)$ is $K=0$, and this means that the point $\gamma(s)$ is either parabolic or planar.
\end{proof}

%第三題
\setcounter{theProblemCounter}{2}
\begin{problem}[Ex P151 3]
\begin{enumerate}
\item[]
\item[(a)] Let $C\subset S$ be a regular curve on a surface $S$ with Gaussian curvature $K > 0$. Show that the curvature $\kappa$ of $C$ at $p$ satisfies \[ \kappa\ge \min(|\kappa_1|, |\kappa_2|),\] where $\kappa_1, \kappa_2$ are the principal curvatures of $S$ at $p$.
\item[(b)] 為什麼上一小題需要 $\kappa>0$ 的條件,$\kappa\ge 0$ 不可以嗎? %好像的確是想說 \kappa 耶(?)
\end{enumerate}
\end{problem}
\begin{proof}
\begin{enumerate}
\item[(a)]
\begin{align*}
\kappa&\ge |\kappa_n|\\
&=|\kappa_1\cos^2\theta+\kappa_2\sin^2\theta|\\
&=|\kappa_1|\cos^2\theta+|\kappa_2|\sin^2\theta (\because\kappa_1, \kappa_2\textrm{ has equal sign.})\\
&\ge \min(|\kappa_1|, |\kappa_2|)(\cos^2\theta+\sin^2\theta)\\
&=\min(|\kappa_1|, |\kappa_2|)
\end{align*}
\item[(b)]
% 若 $K=0$, 則 $\kappa_1, \kappa_2$ 中必定要有一個為零, 則 $\min(|\kappa_1|, |\kappa_2|)=0$, 而 $\kappa\ge 0$ 顯然成立. 故條件亦可改為 $K\ge 0$.
\end{enumerate}
\end{proof}

%第七題
\setcounter{theProblemCounter}{6}
\begin{problem}
\begin{enumerate}
\item[]
\item[(a)] $T_\lambda$ 是縮放 $\lambda$ 倍的映射,$\lambda>0$。$\mathbb{X}:\Omega\to \mathbb{R}^3$ regular surface。討論 $T_\lambda\circ \mathbb{X}:\Omega\to\mathbb{R}^3$ 上對應點 $\kappa_n, H, K$ 的變化。
\item[(b)] $\mathbb{X}:\begin{array}{c}\Omega\\(u,v)\end{array}\to \mathbb{R}^3$,若定義 $\overline{\mathbb{X}}(u, v)= \mathbb{X}(v, u)$(因此 $N$ 轉向)。討論 $\overline{\mathbb{X}}(\Omega)$ 上相對應點的 $K_n, H, K$ 變化。
\end{enumerate}
\end{problem}
\begin{proof}
\end{proof}

%第九題
\setcounter{theProblemCounter}{8}
\begin{problem}[旋轉面]
$\mathbb{X}(u, v)=(f(u)\cos v, f(u)\sin v, g(u))$,$f>0$ \begin{enumerate}
\item[(a)] 計算其 $e, f, g, H, K$
\item[(b)] 討論其 principal direction 與 principal curvature $K_1, K_2$。
\end{enumerate}
\end{problem}
\begin{proof}
To avoid the notational ambiguity, let $\mathbb{X}(u, v)=(s(u)\cos v, s(u)\sin v, t(u))$, and that $s>0$. \begin{enumerate}
\item[(a)] We have
\begin{align*}
\mathbb{X}_u =& (s'(u)\cos v, s'(u)\sin v, t'(u)); \\
\mathbb{X}_v =& (-s(u)\sin v, s(u)\cos v, 0); \\
E =& \langle \mathbb{X}_u, \mathbb{X}_u\rangle = s'(u)^2 + t'(u)^2 \\
F =& \langle \mathbb{X}_u, \mathbb{X}_v\rangle = 0 \\
G =& \langle \mathbb{X}_v, \mathbb{X}_v\rangle = s(u)^2 \\
\mathbb{X}_{uu} =& (s''(u)\cos v, s''(u)\sin v, t''(u)); \\
\mathbb{X}_{uv} =& (-s'(u)\sin v, s'(u)\cos v, 0); \\
\mathbb{X}_{vv} =& (-s(u)\cos v, -s(u)\sin v, 0); \\
N =& \frac{\mathbb{X}_u \times \mathbb{X}_v}{|\mathbb{X}_u \times \mathbb{X}_v|} = \frac{(-t'(u)s(u)\cos v, -t'(u)s(u)\sin v, s'(u)s(u))}{\sqrt{t'(u)^2s(u)^2 + s'(u)^2s(u)^2}} \\
  =& \frac{(-t'(u)\cos v, -t'(u)\sin v, s'(u))}{\sqrt{t'(u)^2 + s'(u)^2}}; \\
e =& \langle N, \mathbb{X}_{uu}\rangle = \frac{-s''(u)t'(u) + t''(u)s'(u)}{\sqrt{t'(u)^2 + s'(u)^2}} \\
f =& \langle N, \mathbb{X}_{uv}\rangle = 0 \\
g =& \langle N, \mathbb{X}_{vv}\rangle = \frac{s(u)t'(u)}{\sqrt{t'(u)^2 + s'(u)^2}} \\
-dN^T =& \left[\begin{array}{cc}e & f \\ f & g\end{array}\right]\left[\begin{array}{cc}E & F \\ F & G\end{array}\right]^{-1} = \frac 1{EG - F^2}\left[\begin{array}{cc}e & f \\ f & g\end{array}\right]\left[\begin{array}{cc}G & -F \\ -F & E\end{array}\right] = \left[\begin{array}{cc}e/E & 0 \\ 0 & g/G\end{array}\right]\\
K =& \det(-dN) = \frac{eg}{EG} \\
H =& \frac 12\text{tr}(-dN) = \frac{eG + gE}{2EG}
\end{align*}
\item[(b)]
Since $-dN$ is already a diagonal matrix, clearly, \begin{align*}
K_1 =& e/E; \\
K_2 =& g/G; \\
V_1 =& \mathbb{X}_u; \\
V_2 =& \mathbb{X}_v;
\end{align*}
\end{enumerate}
\end{proof}

%第十題
\setcounter{theProblemCounter}{9}
\begin{problem}[管面]
$\mathbb{X}(s,\theta) = \gamma(s)+\cos\theta\vec{n}(s) + \sin\theta\vec{b}(s)$, $0<\kappa < 1$
\begin{enumerate}
\item[(a)] 計算其 $e, f, g, H, K$
\item[(b)] 討論曲面上 $K$ 的分佈。
\end{enumerate}
\end{problem}
\begin{proof}
\begin{enumerate}
\item[(a)] 
Let $\vec{t}(s), \vec{n}(s), \vec{b}(s)$ be the basis, and $\gamma(0)$ be the origin.
\begin{align*}
\mathbb{X}_{s} =& (1 - \kappa\cos\theta, \tau\sin\theta, -\tau\cos\theta); \\
\mathbb{X}_{\theta} =& (0, -\sin\theta, \cos\theta); \\
N =& \frac{\mathbb{X}_{s} \times \mathbb{X}_{\theta}}{|\mathbb{X}_{s} \times \mathbb{X}_{\theta}|} = \frac{(0, (\kappa\cos\theta - 1)\cos\theta, (\kappa\cos\theta - 1)\sin\theta)}{|\kappa\cos\theta - 1|} \\
  =& (0, -\cos\theta, -\sin\theta); \quad\text{(since $\kappa< 1$)} \\
\mathbb{X}_{ss} =& (-\sin\theta\kappa\tau, \kappa + \cos\theta(-\kappa^2 - \tau^2), -\sin\theta\tau^2); \\
\mathbb{X}_{s\theta} =& (1 + \kappa\sin\theta, \tau\cos\theta, \tau\sin\theta); \\
\mathbb{X}_{\theta\theta} =& (0, -\cos\theta, -\sin\theta); \\
e =& -\kappa\cos\theta + \cos^2\theta(\kappa^2 + \tau^2) + \tau^2\sin^2\theta = -\kappa\cos\theta + \kappa^2\cos^2\theta + \tau^2; \\
f =& -\tau; \\
g =& 1; \\
dN(\mathbb{X}_{s}) =& N_{s} = (0, 0, 0); \\
dN(\mathbb{X}_{\theta}) =& N_{\theta} = (0, \sin\theta, -\cos\theta) = -\mathbb{X}_{\theta}; \\
\kappa_1 =& 0; \\
\kappa_2 =& 1; \\
K =& 0; \\
H =& 1/2.
\end{align*}
\item[(b)] 
\begin{align*}
K = 0 \text{everywhere}
\end{align*}
\end{enumerate}
\end{proof}
\end{document}
