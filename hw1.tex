\documentclass[12pt,a4paper]{article}
\usepackage{amsmath, amssymb, amsthm}
%加這個就可以設定字體
\usepackage{fontspec}
\usepackage{xkeyval} %MikTeX 2.9 版本相容有誤, 以此修正
%使用xeCJK,其他的還有CJK或是xCJK
\usepackage{xeCJK}
%設定英文字型,不設的話就會使用預設的字型
%\setmainfont{Times New Roman}
%設定中英文的字型
%字型的設定可以使用系統內的字型,而不用像以前一樣另外安裝
\setCJKmainfont{LiHei Pro}
%中文自動換行
\XeTeXlinebreaklocale "zh"
%文字的彈性間距
\XeTeXlinebreakskip = 0pt plus 1pt
%設定段落之間的距離
\setlength{\parskip}{0.3cm}
%設定行距
%\linespread{1.5}\selectfont

%先暫時用水泥字型,如果任何人看不下去就改吧XD
\usepackage[T1]{fontenc}
\usepackage{concmath}

\newcounter{theProblemCounter}
\newtheorem{problem}[theProblemCounter]{Problem}


\begin{document}
\title{Geometry Homework 1}
\author{{\it{B96201044}} 黃上恩, {\it{B98901182}} 時丕勳, {\it{K0020100x}} 劉士瑋}
\date{\today}
\maketitle

%第三題
\setcounter{theProblemCounter}{2}
\begin{problem}[P7: 4]
Let $\alpha:(0, \pi)\to \mathbf{R}^2$ be given by
\[ \alpha(t) = \left(\cos t, \cos t + \log\tan\frac{t}{2}\right),\]
where $t$ is the angle that the $y$ axis makes with the vector $\alpha(t)$. The trace of $\alpha$ is called the \emph{tractrix} (Fig. 1-9). Show that
\begin{enumerate}
\item[(a)] $\alpha$ is a differentiable parametrized curve, regular except at $t=\pi/2$.
\item[(b)] The length of the segment of the tangent of the tractrix between the point of tangency and the $y$ axis is constantly equal to $1$.
\end{enumerate}
\end{problem}

\begin{proof}
This is the proof.
\end{proof}

%第五題
\setcounter{theProblemCounter}{4}
\begin{problem}[P47: 6]
If a closed plane curve $C$ is contained inside a disk of radius $r$, prove that there exists a point $p\in C$ such that the curvature $\kappa$ of $C$ at $p$ satisfies $|\kappa|\ge 1/r$.
\end{problem}

%第八題
\setcounter{theProblemCounter}{7}
\begin{problem}[Curvature is a geometric object I.]
$X(s)=(x(s), y(s))$, where $s$ is the arc-length parameter.
\[ M = \left[
\begin{array}{cc} a_{11} & a_{12} \\ a_{21} & a_{22} \end{array},
\right]
M^t = M^{-1}, \mbox{i.e. $M$ is orthogonal.}
\]
Let $\bar{M}(s) = M\cdot \left[\begin{array}{c} x(s)\\y(s)\end{array}\right] + \left[\begin{array}{c}\alpha \\ \beta\end{array}\right]$,  $\alpha, \beta\in \mathbf{R}$. What is the relation between $\kappa_X(s)$ and $\kappa_{\bar{X}}(s)$?
\end{problem}

%第九題
\begin{problem}[Curvature is a geometric object II.]
$X(t) = (x(t), y(t))$ be a regular curve. Let
\[ \kappa(x(t), y(t)) \equiv \kappa(t) = \frac{
\left|
\begin{array}{cc} x' & y' \\ x'' & y'' \end{array}
\right|
}{(x'^2+y'^2)^\frac32} \]
Let $Y(u) = X(t(u))$, $t'(u)\ne 0$. Discuss the relation of $\kappa(x(t), y(t))$ and $\kappa(x(t(u)), y(t(u)))$ at the corresponding points.
\end{problem}

%第十題
\begin{problem}
Let $F(x, y)=c$ defines a plane curve. Prove that the curvature of the curve satisfies
\[
|\kappa| = \left|
\frac{
\left[
\begin{array}{cc} F_y, & -F_x\end{array}
\right]
\left[
\begin{array}{cc} F_{xx} & F_{xy} \\ F_{xy} & F_{yy} \end{array}
\right]
\left[
\begin{array}{c} F_y \\ -F_x\end{array}
\right]
}{(F_x^2+F_y^2)^\frac32}
\right|
\]
Where $F_x^2+F_y^2\ne 0$.
\end{problem}

\end{document}