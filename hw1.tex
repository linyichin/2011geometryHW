\documentclass[12pt,a4paper]{article}
\usepackage{amsmath, amssymb, amsthm}
%加這個就可以設定字體
\usepackage{fontspec}
\usepackage{xkeyval} %MikTeX 2.9 版本相容有誤, 以此修正
%使用xeCJK,其他的還有CJK或是xCJK
\usepackage{xeCJK}
%設定英文字型,不設的話就會使用預設的字型
%\setmainfont{Times New Roman}
%設定中英文的字型
%字型的設定可以使用系統內的字型,而不用像以前一樣另外安裝
\setCJKmainfont{LiHei Pro}
%中文自動換行
\XeTeXlinebreaklocale "zh"
%文字的彈性間距
\XeTeXlinebreakskip = 0pt plus 1pt
%設定段落之間的距離
\setlength{\parskip}{0.3cm}
%設定行距
%\linespread{1.5}\selectfont

%先暫時用水泥字型,如果任何人看不下去就改吧XD
\usepackage[T1]{fontenc}
\usepackage{concmath}

\newcounter{theProblemCounter}
\newtheorem{problem}[theProblemCounter]{Problem}


\begin{document}
\title{Geometry Homework 1}
\author{{\it{B96201044}} 黃上恩, {\it{B98901182}} 時丕勳, {\it{K0020100x}} 劉士瑋}
\date{\today}
\maketitle

%第三題
\setcounter{theProblemCounter}{2}
\begin{problem}[P7: 4]
Let $\alpha:(0, \pi)\to \mathbf{R}^2$ be given by
\[ \alpha(t) = \left(\sin t, \cos t + \log\tan\frac{t}{2}\right),\] % 題目錯了... //是說...錯在哪裡呀OQ 我完全看不出來...
where $t$ is the angle that the $y$ axis makes with the vector $\alpha(t)$. The trace of $\alpha$ is called the \emph{tractrix} (Fig. 1-9). Show that
\begin{enumerate}
\item[(a)] $\alpha$ is a differentiable parametrized curve, regular except at $t=\pi/2$.
\item[(b)] The length of the segment of the tangent of the tractrix between the point of tangency and the $y$ axis is constantly equal to $1$.
\end{enumerate}
\end{problem}

\begin{proof} % 證明先打起來, 格式卡恩覺得要怎麼弄比較好呢?

\begin{enumerate}
\item[]
\item[(a)] Let $x(t)=\sin t$, $y(t)=\cos t+\log\tan\frac{t}{2}$, then \[x'(t)=\cos t; \ \  y'(t)=-\sin t+\frac{1}{\sin t}.\]

It's trivial that both $x'(t)$ and $y'(t)$ are infinitely differentiable in $(0,\pi)$, so $\alpha$ is a differentiable parametrized curve.

$x'(t)=0, y'(t)=0 \Longleftrightarrow t=\frac{\pi}{2}$, so $\alpha$ is regular except at $t=\pi/2$.
\item[(b)] The intersection of $y$ axis and the tangent of the tractrix is $\left(0,y(t)-\frac{y'(t)}{x'(t)}x(t)\right)$.

The length of the segment of the tangent of the tractrix between the point of tangency and the $y$ axis is $\sqrt{x(t)^2+\left(\frac{y'(t)}{x'(t)}x(t)\right)^2}$

\begin{math}\begin{aligned}
x(t)^2+\left(\frac{y'(t)}{x'(t)}x(t)\right)^2 &= \sin^2{t}\left(1+\left(\frac{y'(t)}{x'(t)}\right)^2\right)\\
&= \sin^2{t}\left(1+\left(\frac{-\sin t+\frac{1}{\sin t}}{\cos t}\right)^2\right)\\
&= \sin^2{t}\left(1+\left(\frac{1-\sin^2 t}{\sin{t}\cos{t}}\right)^2\right)\\
&= \sin^2{t}\left(\frac{1}{\sin^2{t}}\right)\\
&= 1
\end{aligned}\end{math}

So the length of the segment of the tangent of the tractrix between the point of tangency and the $y$ axis $ = \sqrt{x(t)^2+\left(\frac{y'(t)}{x'(t)}x(t)\right)^2} = 1$.
\end{enumerate}
\end{proof}

%第五題 %皮皮 電波 對不起我打錯題目了...
\setcounter{theProblemCounter}{4}
\begin{problem}[P47: 6]
Let $\alpha(s), s\in [0, l]$ be a closed convex plane curve positively oriented. The curve
\[ \beta(s)=\alpha(s) - rn(s),\]
where $r$ is a positive constant and $n$ is the normal vector, is called a \emph{parallel} curve to $\alpha$ (Fig. 1-37). Show that
\begin{enumerate}
\item[(a)] Length of $\beta = $ length of $\alpha + 2\pi r$.
\item[(b)] $A(\beta)=A(\alpha)+rl+\pi r^2$.
\item[(c)] $\kappa_\beta(s) = \kappa_\alpha(s)/(1+r\kappa_\alpha(s))$.
\end{enumerate}
For (a)-(c), $A(\cdot)$ denotes the area bounded by the corresponding curve, and $\kappa_\alpha, \kappa_\beta$ are the curvatures of $\alpha$ and $\beta$, respectively.
\end{problem}
\begin{proof}
%Let $p\in C$ is the farthest point from the center of the disk on the curve. We want to prove that $\kappa(p)\ge 1/r$.

%WLOG, we can rotate and translate the curve so that $p$ is at the origin and the tangent of $C$ at $p$ is the $x$ axis.
\end{proof}
%第八題
\setcounter{theProblemCounter}{7}
\begin{problem}[Curvature is a geometric object I.]
$X(s)=(x(s), y(s))$, where $s$ is the arc-length parameter.
\[ M = \left[
\begin{array}{cc} a_{11} & a_{12} \\ a_{21} & a_{22} \end{array},
\right]
M^t = M^{-1}, \mbox{i.e. $M$ is orthogonal.}
\]
Let $\bar{X}(s) = M\cdot \left[\begin{array}{c} x(s)\\y(s)\end{array}\right] + \left[\begin{array}{c}\alpha \\ \beta\end{array}\right]$,  $\alpha, \beta\in \mathbf{R}$. What is the relation between $\kappa_X(s)$ and $\kappa_{\bar{X}}(s)$?
\end{problem}
\begin{proof}
\end{proof}

%第九題
\begin{problem}[Curvature is a geometric object II.]
$X(t) = (x(t), y(t))$ be a regular curve. Let
\[ \kappa(x(t), y(t)) \equiv \kappa(t) = \frac{
\left|
\begin{array}{cc} x' & y' \\ x'' & y'' \end{array}
\right|
}{(x'^2+y'^2)^\frac32} \]
Let $Y(u) = X(t(u))$, $t'(u)\ne 0$. Discuss the relation of $\kappa(x(t), y(t))$ and $\kappa(x(t(u)), y(t(u)))$ at the corresponding points.
\end{problem}
\begin{proof}
\end{proof}

%第十題
\begin{problem}
Let $F(x, y)=c$ defines a plane curve. Prove that the curvature of the curve satisfies
\[
|\kappa| = \left|
\frac{
\left[
\begin{array}{cc} F_y, & -F_x\end{array}
\right]
\left[
\begin{array}{cc} F_{xx} & F_{xy} \\ F_{xy} & F_{yy} \end{array}
\right]
\left[
\begin{array}{c} F_y \\ -F_x\end{array}
\right]
}{(F_x^2+F_y^2)^\frac32}
\right|
\]
Where $F_x^2+F_y^2\ne 0$.
\end{problem}
\begin{proof}
Let $\alpha$ be a point in the plane such that $F(\alpha)=c$. Consider the circle of curvature passing through $\alpha$. If we observed the intersection of two line respectively perpendicular to the lines tangent to $F=c$ and passing respectively through $\alpha$ and another point $\alpha'\in F=c$, the intersection approaphes the centre of the circle $o$ as $\alpha'\rightarrow \alpha$. 
Thus,
$$
	|\alpha' - \alpha| = r\sin \theta
$$
, where $\theta$ is the angle between the vectors $o-\alpha$ and $o-\alpha'$. By the formula of exterior product, 
$$
	\sin\theta = \frac{|(o - \alpha) \times (o - \alpha')|}{|o - \alpha||o - \alpha'|}
$$
Let $n$ denote $ \displaystyle \nabla F/|\nabla F| $ rotated counterclockwise by $\pi/2$.
Since $\alpha' - \alpha$ is perpendicular to $\displaystyle\frac{\nabla F(\alpha)}{|\nabla F(\alpha)|} $, and since $o - \alpha$ and $o - \alpha'$ are respectively parallel to $\displaystyle \frac{\nabla F(\alpha)}{|\nabla F(\alpha)|} $ and $ \displaystyle\frac{\nabla F(\alpha')}{|\nabla F(\alpha')|} $, we obtained
\begin{align*}
	|\kappa| = 1/r = &\lim_{\alpha'\rightarrow \alpha}\frac{|(o - \alpha) \times (o - \alpha')|}{|\alpha' - \alpha||o - \alpha||o - \alpha'|} =
	      \lim_{\alpha'\rightarrow \alpha}\frac{|\nabla F(\alpha) \times \nabla F(\alpha')|}{|\alpha' - \alpha||\nabla F(\alpha)||\nabla F(\alpha')|}\\
	      = &\lim_{t\rightarrow 0}\frac{|\nabla F(\alpha) \times \nabla F(\alpha + tn)|}{tn|\nabla F(\alpha)||\nabla F(\alpha + tn)|} \\
	      = &\lim_{t\rightarrow 0}\frac{|(F_x(\alpha)\vec i + F_y(\alpha)\vec j) \times (F_x(\alpha + tn)\vec i + F_y(\alpha + tn)\vec j)|}{tn|\nabla F(\alpha)|^2}\\
	      = &\lim_{t\rightarrow 0}\frac{|F_x(\alpha)F_y(\alpha + tn) - F_x(\alpha + tn)F_y(\alpha)|}{tn|\nabla F(\alpha)|^2}\\
	      = &\lim_{t\rightarrow 0}\frac{|F_x(\alpha)[F_y(\alpha + tn) - F_y(\alpha)] - [F_x(\alpha + tn) - F_x(\alpha)]F_y(\alpha)|}{tn|\nabla F(\alpha)|^2}\\
	      = &\frac{|F_x(F_{yx}\vec i + F_{yy}\vec j)\cdot n - F_y(F_{xx}\vec i + F_{xy}\vec j)\cdot n|}{|\nabla F|^2}\\
	      = &\frac{|[(F_xF_{yx} - F_yF_{xx})\vec i + (F_xF_{yy} - F_yF_{xy})\vec j]\cdot n|}{|\nabla F|^2}\\
	      = &\frac{|[(F_xF_{yx} - F_yF_{xx})\vec i + (F_xF_{yy} - F_yF_{xy})\vec j]\cdot (F_y\vec i - F_x\vec j)/|\nabla F||}{|\nabla F|^2}\\
	      = &
	      \frac{
		      \left|
		      \left[
			      \begin{array}{cc} F_y & -F_x\end{array}
		      \right]
			      \left[
			      \begin{array}{cc} F_{xx} & F_{xy} \\ F_{xy} & F_{yy} \end{array}
		      \right]
			      \left[
			      \begin{array}{c} F_y \\ -F_x\end{array}
		      \right]
		      \right|
	      }
		{
			|\nabla F|^3
		}\\
		= &
		\left|
		\frac{
			\left[
				\begin{array}{cc} F_y, & -F_x\end{array}
			\right]
				\left[
				\begin{array}{cc} F_{xx} & F_{xy} \\ F_{xy} & F_{yy} \end{array}
			\right]
				\left[
				\begin{array}{c} F_y \\ -F_x\end{array}
			\right]
		}{(F_x^2+F_y^2)^\frac32}
		\right|
\end{align*}
\end{proof}

\end{document}
