\documentclass[10pt,a4paper]{article}
\usepackage{amsmath, amssymb, amsthm}
%加這個就可以設定字體
\usepackage{fontspec}
\usepackage{xkeyval} %MikTeX 2.9 版本相容有誤, 以此修正
%使用xeCJK,其他的還有CJK或是xCJK
\usepackage{xeCJK}
%設定英文字型,不設的話就會使用預設的字型
%\setmainfont{Times New Roman}
%設定中英文的字型
%字型的設定可以使用系統內的字型,而不用像以前一樣另外安裝
%\setCJKmainfont{文泉驛微米黑}
\setCJKmainfont{WenQuanYi Micro Hei}
\setCJKfamilyfont{lh}{LiHei Pro}
\newcommand{\LiHei}{\CJKfamily{lh}}
%中文自動換行
\XeTeXlinebreaklocale "zh"
%文字的彈性間距
\XeTeXlinebreakskip = 0pt plus 1pt
%設定段落之間的距離
\setlength{\parskip}{0.3cm}
%設定行距
%\linespread{1.5}\selectfont
\usepackage{enumerate}

%先暫時用水泥字型,如果任何人看不下去就改吧XD
\usepackage[T1]{fontenc}
\usepackage{concmath}
%\usepackage{mathptmx} %Times

\newcounter{theProblemCounter}
\newtheorem{problem}[theProblemCounter]{Problem}

\begin{document}
\title{{\fontspec{Copperplate Gothic Bold}Geometry Homework 7}}
%\title{{\fontspec{Copperplate}Geometry Homework 5}}
\author{{\it{B96201044}} {\LiHei 黃上恩}, {\it{B98901182}} {\LiHei 時丕勳}, {\it{K0020100x}} {\LiHei 劉士瑋}}
\date{\today}
\maketitle

\newcommand{\bx}{\mathbb{X}}
%第二題
\setcounter{theProblemCounter}{1}
\begin{problem}
若 $F(x, y, z)=0$ 定義一 surface,證明 $\nabla F\neq 0$ 的地方 Gauss curvature $K=\frac{\nabla F^t A\nabla F}{\|\nabla F\|^4}$。其中 $A$ 為 $\partial^2 F=\left(
\begin{array}{ccc}
F_{xx} & F_{xy} & F_{xz} \\
F_{yx} & F_{yy} & F_{yz} \\
F_{zx} & F_{zy} & F_{zz}\end{array}
\right)$ 的 adjoint Matrix, i.e. $A=\det(\partial^2F)(\partial^2F)^{-1}$ % f 應該是打錯吧
\end{problem}
\begin{proof}
因為 $K$ 為局部性質,而在 $\nabla F\neq 0$ 的地方我們可以使用隱函數定理將其中一維表示為另兩維的函數,WLOG 不妨設 $z=z(x,y)$ 在某點附近。\\
\begin{align*}
F(x,y,z(x,y))&=0\\
\bx(x,y)&=(x,y,z(x,y))\\
\rightarrow \bx_x&=(1,0,z_x)\\
\bx_y&=(0,1,z_y)\\
\rightarrow N&=\frac{\bx_x\times\bx_y}{|\bx_x\times\bx_y|}\\
&=\frac{(-z_x,-z_y,1)}{\sqrt{1+z_x^2+z_y^2}}\\
\end{align*}
\begin{align*}
E&=\langle\bx_x, \bx_x\rangle\\
&=1+z_x^2\\
F&=\langle\bx_x, \bx_y\rangle\\
&=z_xz_y\\
G&=\langle\bx_y, \bx_y\rangle\\
&=1+z_y^2\\
\end{align*}
\begin{align*}
\bx_{xx}&=(0,0,z_{xx})\\
\bx_{xy}&=(0,0,z_{xy})\\
\bx_{yy}&=(0,0,z_{yy})\\
\rightarrow e&=\langle N, \bx_{xx}\rangle\\
&=\frac{z_{xx}}{\sqrt{1+z_x^2+z_y^2}}\\
f&=\langle N, \bx_{xy}\rangle\\
&=\frac{z_{xy}}{\sqrt{1+z_x^2+z_y^2}}\\
g&=\langle N, \bx_{yy}\rangle\\
&=\frac{z_{yy}}{\sqrt{1+z_x^2+z_y^2}}\\
\rightarrow K&=det([-dN])\\
&=det\left(\begin{bmatrix}E&F\\F&G\end{bmatrix}^{-1}\begin{bmatrix}e&f\\f&g\end{bmatrix}\right)\\
&=det\left(\begin{bmatrix}E&F\\F&G\end{bmatrix}\right)^{-1}\det\left(\begin{bmatrix}e&f\\f&g\end{bmatrix}\right)\\
&=\frac{z_{xx}z_{yy}-z_{xy}^2}{(1+z_x^2+z_y^2)^2}
\end{align*}
\begin{align*}
\frac{\partial F(x,y,z)}{\partial x}&=0\\
&=F_x(x,y,z)+F_z(x,y,z)z_x\\
\rightarrow z_x&=-\frac{F_x}{F_z}\\
\frac{\partial F(x,y,z)}{\partial y}&=0\\
&=F_y(x,y,z)+F_z(x,y,z)z_y\\
\rightarrow z_y&=-\frac{F_y}{F_z}\\
\end{align*}
\begin{align*}
\frac{\partial^2 F(x,y,z)}{\partial x^2}&=0\\
&=\frac{\partial}{\partial x}(F_x(x,y,z)+F_z(x,y,z)z_x)\\
&=F_{xx}(x,y,z)+2F_{xz}(x,y,z)z_x+F_{zz}(x,y,z)z_x+F_z(x,y,z)z_{xx}\\
\rightarrow z_{xx}&=-\frac{F_{xx}-(2F_{xz}+F_{zz})\frac{F_x}{F_z}}{F_z}\\
\frac{\partial^2 F(x,y,z)}{\partial y^2}&=0\\
&=\frac{\partial}{\partial y}(F_y(x,y,z)+F_z(x,y,z)z_y)\\
&=F_{yy}(x,y,z)+2F_{yz}(x,y,z)z_y+F_{zz}(x,y,z)z_y+F_z(x,y,z)z_{yy}\\
\rightarrow z_{yy}&=-\frac{F_{yy}-(2F_{yz}+F_{zz})\frac{F_y}{F_z}}{F_z}\\
\frac{\partial^2 F(x,y,z)}{\partial x\partial y}&=0\\
&=\frac{\partial}{\partial x}(F_y(x,y,z)+F_z(x,y,z)z_y)\\
&=F_{xy}(x,y,z)+F_{yz}(x,y,z)z_x+F_{xz}(x,y,z)z_y+F_{zz}(x,y,z)z_xz_y+F_z(x,y,z)z_{xy}\\
\rightarrow z_{xy}&=-\frac{F_{xy}-F_{yz}\frac{F_x}{F_z}-F_{xz}\frac{F_y}{F_z}}{F_z}\\
\end{align*}
\begin{align*}
\rightarrow K&=\frac{z_{xx}z_{yy}-z_{xy}^2}{(1+z_x^2+z_y^2)^2}\\
&=\frac{\left(\frac{F_{xx}-(2F_{xz}+F_{zz})\frac{F_x}{F_z}}{F_z}\right)\left(\frac{F_{yy}-(2F_{yz}+F_{zz})\frac{F_y}{F_z}}{F_z}\right)-\left(\frac{F_{xy}-F_{yz}\frac{F_x}{F_z}-F_{xz}\frac{F_y}{F_z}}{F_z}\right)^2}{(1+\left(\frac{F_x}{F_z}\right)^2+\left(\frac{F_y}{F_z}\right)^2)^2}\\
&=\frac{\left(F_{xx}F_z-(2F_{xz}+F_{zz})F_x\right)\left(F_{yy}F_z-(2F_{yz}+F_{zz})F_y\right)-\left(F_{xy}F_z-F_{yz}F_x-F_{xz}F_y\right)^2}{(F_z^2+F_x^2+F_y^2)^2}\\
\end{align*}
\end{proof}
%第三題
\setcounter{theProblemCounter}{2}
\begin{problem}[Ex P168 4]
Determine the asymptotic curves and the lines of curvature of $z=xy$.
\end{problem}
\begin{proof}
\end{proof}

%第四題
\setcounter{theProblemCounter}{3}
\begin{problem}
已知 $\mathbb{X}(u, v)$ 為一 surface $\subset \mathbb{R}^3$ 且 $E=G=(1+u^2+v^2)^2, F=0$ 而且 $e=1, f=\sqrt{3}, g=-1$
\begin{enumerate}[{ (}a{)}]
\item 求在 $\mathbb{X}(1, 1)$ 的 $K$ 與 $H$
\item 如何決定過 $\mathbb{X}(1, 1)$ 的 line of curvature 與 asymptotic curve(如果有的話)
\end{enumerate}
\end{problem}
\begin{proof}
\begin{enumerate}[{ (}a{)}]
\item
at $(1,1)$, $E=G=9, F=0$.\\
\begin{align*}
[-dN]&=\begin{bmatrix}E&F\\F&G\end{bmatrix}^{-1}\begin{bmatrix}e&f\\f&g\end{bmatrix}\\
&=\begin{bmatrix}9&0\\0&9\end{bmatrix}^{-1}\begin{bmatrix}1&\sqrt{3}\\\sqrt{3}&-1\end{bmatrix}\\
&=\begin{bmatrix}\frac{1}{9}&\frac{1}{3\sqrt{3}}\\\frac{1}{3\sqrt{3}}&-\frac{1}{9}\end{bmatrix}
\end{align*}
So $K=\det([-dN])=-\frac{4}{81}$, $H=\textrm{tr}([-dN])=0$.
\end{enumerate}
\end{proof}

%第五題
\setcounter{theProblemCounter}{4}
\begin{problem}
$\mathbb{X}(u, v) = (v\cos u, v\sin u, u)$,令 $\gamma(t)=\mathbb{X}(t, 1)$
\begin{enumerate}[{ (}a{)}]
\item 求 $\gamma(t)$ 的 $\kappa_n, \kappa_g, \tau_g$
\item 與 $\gamma(t)$ 的 $\kappa, \tau$ 有何關係
\end{enumerate}
\end{problem}
\begin{proof}
\end{proof}

%第六題
\setcounter{theProblemCounter}{5}
\begin{problem}
令 $(x(t), y(t)) = (t-\mathrm{tanh\ } t, \mathrm{sech\ } t)$ 這基本就是 p7(4) 的 tratrix
\begin{enumerate}[{ (}a{)}]
\item 將此曲線化作長度參數
\item 利用上小題,求此曲線繞 $x$ 軸旋轉的旋轉體的 $K$
\end{enumerate}
\end{problem}

\begin{proof}
\end{proof}

\end{document}
