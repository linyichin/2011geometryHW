\documentclass[10pt,a4paper]{article}
\usepackage{amsmath, amssymb, amsthm}
%加這個就可以設定字體
\usepackage{fontspec}
\usepackage{xkeyval} %MikTeX 2.9 版本相容有誤, 以此修正
%使用xeCJK,其他的還有CJK或是xCJK
\usepackage{xeCJK}
%設定英文字型,不設的話就會使用預設的字型
%\setmainfont{Times New Roman}
%設定中英文的字型
%字型的設定可以使用系統內的字型,而不用像以前一樣另外安裝
%\setCJKmainfont{文泉驛微米黑}
\setCJKmainfont{WenQuanYi Micro Hei}
\setCJKfamilyfont{lh}{LiHei Pro}
\newcommand{\LiHei}{\CJKfamily{lh}}
%中文自動換行
\XeTeXlinebreaklocale "zh"
%文字的彈性間距
\XeTeXlinebreakskip = 0pt plus 1pt
%設定段落之間的距離
\setlength{\parskip}{0.3cm}
%設定行距
%\linespread{1.5}\selectfont

%先暫時用水泥字型,如果任何人看不下去就改吧XD
\usepackage[T1]{fontenc}
\usepackage{concmath}
%\usepackage{mathptmx} %Times

\newcounter{theProblemCounter}
\newtheorem{problem}[theProblemCounter]{Problem}

\begin{document}
\title{{\fontspec{Copperplate Gothic Bold}Geometry Homework 5}}
%\title{{\fontspec{Copperplate}Geometry Homework 5}}
\author{{\it{B96201044}} {\LiHei 黃上恩}, {\it{B98901182}} {\LiHei 時丕勳}, {\it{K0020100x}} {\LiHei 劉士瑋}}
\date{\today}
\maketitle

\newcommand{\bx}{\mathbb{X}}
%第一題
\setcounter{theProblemCounter}{0}
\begin{problem}[參見 P67 Ex16]
考慮
\[
\begin{array}{ccccc}
\bx: & \mathbb{R}^2 & \to & S^2\setminus \{N\} & , N=(0,0,1) \\
& (u, v) & \mapsto & \left(\frac{2u}{u^2+v^2+1}, \frac{2v}{u^2+v^2+1},\frac{u^2+v^2-1}{u^2+v^2+1}\right) &
\end{array}
\]

\begin{enumerate}
\item[(a)] 檢查這的確是 $S^2\setminus \{N\}$ 的參數式
\item[(b)] 計算 $E, F, G$,$E=G$ 嗎?
\item[(c)] 計算 $\bx_u, \bx_v$
\item[(d)] 若 $W_1, W_2$ 是 $\mathbb{R}^2$ 兩以 $a$ 為起點的向量,說明 $W_1, W_2$ 的夾角 $=d\bx(W_1)$ 與 $d\bx(W_2)$ 的夾角
\end{enumerate}
\end{problem}
\begin{proof}
\begin{enumerate}
\item[]
\item[(a)] 考慮 $S^2\setminus \{N\}$ 上面的點 $(x, y, z)$,它必須滿足 $x^2+y^2+z^2=1$ 且 $z\ne 1$。則令 $u=\frac{x}{1-z}, v=\frac{y}{1-z}$,於是有 $\frac{2u}{u^2+v^2+1}=2\frac{x}{1-z}\cdot\frac{(1-z)^2}{x^2+y^2+(1-z)^2} = 2\frac{x}{1-z}\cdot\frac{(1-z)^2}{2-2z} = x$,類似地 $\frac{2v}{u^2+v^2+1}=2\frac{y}{1-z}\cdot\frac{(1-z)^2}{x^2+y^2+(1-z)^2}=y$,$\frac{u^2+v^2-1}{u^2+v^2+1}=\frac{1-z^2-(1-z)^2}{1-z^2+(1-z)^2} = \frac{2z-2z^2}{2-2z} = z$。因此這的確是 $S^2\setminus\{N\}$ 的參數式。
\item[(c)]
$
\mathbb{X}_u = \left(
\frac{2}{u^2+v^2+1}-\frac{4u^2}{(u^2+v^2+1)^2},
-\frac{4uv}{(u^2+v^2+1)^2},
\frac{2u}{u^2+v^2+1}-\frac{2u(u^2+v^2-1)}{(u^2+v^2+1)^2}
\right)
$\\
$
\mathbb{X}_v = \left(
-\frac{4uv}{u^2+v^2+1},
\frac{2}{u^2+v^2+1}-\frac{4v^2}{(u^2+v^2+1)^2},
\frac{2v}{u^2+v^2+1}-\frac{2v(u^2+v^2-1)}{(u^2+v^2+1)^2}
\right)
$
 \item[(b)] 
$
E=\left\langle \mathbb{X}_u, \mathbb{X}_u
\right\rangle = \frac{4}{(u^2+v^2+1)^4}(((u^2+v^2+1)-2u^2)^2 + (2uv)^2 + (u(u^2+v^2+1) - u(u^2+v^2-1))^2) = \frac{4}{(u^2+v^2+1)^4}((v^2-u^2+1)^2+4u^2v^2+4u^2) = \frac{4}{(u^2+v^2+1)^2}
$\\
\\
$
F=\left\langle \mathbb{X}_u, \mathbb{X}_v
\right\rangle = \frac{4}{(u^2+v^2+1)^4}
(-2uv(v^2-u^2+1) -2uv(u^2-v^2+1) + 4uv)
= \frac{4}{(u^2+v^2+1)^4}\cdot 0 = 0
$\\
\\
$
G=\left\langle \mathbb{X}_v, \mathbb{X}_v
\right\rangle = \frac{4}{(u^2+v^2+1)^4}
((2uv)^2 + ((u^2+v^2+1)-2v^2)^2 + (v(u^2+v^2+1)-v(u^2+v^2-1))^2)
=\frac{4}{(u^2+v^2+1)^4}((u^2-v^2+1)^2+4u^2v^2+4v^2) = \frac{4}{(u^2+v^2+1)^2} = E$

\item[(d)]
Let $a=(u, v)$. Then $\|d\mathbb{X}(W_1)\|^2=W_1^T
\left[\begin{array}{cc} E & F \\ F & G \end{array}\right] W_1
= \frac{4}{(u^2+v^2+1)^2} \|W_1\|^2$. Similarly $\|d\mathbb{X}(W_2)\| = \frac{4}{(u^2+v^2+1)^2}  \|W_2\|^2$. Now the cosine of angle between them is 
\begin{align*}
\cos\theta &= \left\langle d\mathbb{X}(W_1), d\mathbb{X}(W_2)\right\rangle/(\frac{4}{(u^2+v^2+1)^2}\cdot\|W_1\|\cdot\|W_2\|) \\
&= \left( 
W_1^T\left[\begin{array}{cc} E & F \\ F & G\end{array}
\right] W_2\right) /
\left(\frac{4}{(u^2+v^2+1)^2}\cdot\|W_1\|\cdot\|W_2\|\right)
\\
&= \left\langle W_1, W_2\right\rangle / (\|W_1\|\cdot\|W_2\|)
\end{align*}
which is cosine of angle between $W_1$ and $W_2$.
\end{enumerate}
\end{proof}

%第二題
\setcounter{theProblemCounter}{1}
\begin{problem}[旋轉面]
$\bx(\theta, s)=(a(s)\cos\theta, a(s)\sin\theta, b(s))$,其中 $(a(s), b(s))$ 為長度參數之平面曲線。計算 $E, F, G$ 並討論其 regular 的條件。
\end{problem}
\begin{proof}
\begin{align*}
\bx_{\theta}&=(-a(s)\sin\theta, a(s)\cos\theta,0)\\
\bx_{s}&=(a'(s)\cos\theta, a'(s)\sin\theta, b'(s))\\
\Rightarrow E&=a(s)^2\sin^2\theta+a(s)^2\cos^2\theta\\
&=a(s)^2\\
F&=-a(s)a'(s)\sin\theta\cos\theta+a(s)a'(s)\cos\theta\sin\theta\\
&=0\\
G&=a'(s)^2\sin^2\theta+a'(s)^2\cos^2\theta\\
&=a'(s)^2\\
&=1\\
\left|\bx_{\theta}\times\bx_{s}\right|&=\sqrt{EG-F^2}\\
&=\sqrt{a(s)^2}\\
&=\left|a(s)\right|\\
\end{align*}
So $\bx$ is regular iff $a(s)\neq 0$.
\end{proof}

%第三題
\setcounter{theProblemCounter}{2}
\begin{problem}[管面]
設空間曲線 $\gamma(s)$, $s$ 長度參數,$\vec{t}, \vec{n}, \vec{b}$ 為 Frenet frame。令 $\bx_l(s,\theta)=\gamma(s)+l\cos\theta\vec{n}(s) + l\sin\theta\vec{b}(s), l>0$,計算 $E, F, G$ 並討論其 regular 條件。
\end{problem}
\begin{proof}
\begin{align*}
\bx_{ls}&=\gamma'(s)+l\cos\theta\vec{n}'(s) + l\sin\theta\vec{b}'(s)\\
&=\vec{t}(s)+l\cos\theta\left(-\kappa(s)\vec{t}(s)-\tau(s)\vec{b}(s)\right) + l\tau(s)\sin\theta\vec{n}(s)\\
&=(1-l\kappa(s)\cos\theta)\vec{t}(s) + l\tau(s)\sin\theta\vec{n}(s) - l\tau(s)\cos\theta\vec{b}(s)\\
\bx_{l\theta}&=-l\sin\theta\vec{n}(s) + l\cos\theta\vec{b}(s)\\
\Rightarrow E&=(1-l\kappa(s)\cos\theta)^2+(l\tau(s)\sin\theta)^2+(l\tau(s)\cos\theta)^2\\
&=(1-l\kappa(s)\cos\theta)^2+l^2\tau(s)^2\\
F&=-l^2\tau(s)\sin^2\theta-l^2\tau(s)\cos^2\theta\\
&=-l^2\tau(s)\\
G&=l^2\sin^2\theta+l^2\cos^2\theta\\
&=l^2\\
\left|\bx_{ls}\times\bx_{l\theta}\right|&=\sqrt{EG-F^2}\\
&=\sqrt{l^2\left((1-l\kappa(s)\cos\theta)^2+l^2\tau(s)^2\right)-l^4\tau(s)^2}\\
&=\sqrt{l^2(1-l\kappa(s)\cos\theta)^2}\\
&=l\left|1-l\kappa(s)\cos\theta\right|\\
\end{align*}
So $\bx$ is regualr iff $l\kappa(s)\cos\theta\neq 1 \forall s,\theta$\\
$\Leftrightarrow \frac{1}{l\kappa(s)}\neq \cos\theta$\\
$\Leftrightarrow \left|\frac{1}{l\kappa(s)}\right|>1$\\
$\Leftrightarrow \left|\kappa(s)\right|<\frac{1}{l}$\\
\end{proof}

\newcommand{\mat}[7]{ \left[\begin{array}{cc}#1 & #2\end{array}\right]\left[\begin{array}{cc}#5 & #6 \\ #6 & #7\end{array}\right]\left[\begin{array}{c}#3 \\ #4\end{array}\right] }
\newcommand{\matnl}[7]{ \left[\begin{array}{cc}#1 & #2\end{array}\right]\left[\begin{array}{cc}#5 & #6 \\ #6 & #7\end{array}\right]\\&\left[\begin{array}{c}#3 \\ #4\end{array}\right] }
%第六題
\setcounter{theProblemCounter}{5}
\begin{problem}[Ex6, p100]
Show that
\[ \mathbf{x}(u, v) = (u\sin\alpha\cos v, u\sin\alpha\sin v, u\cos\alpha)
\]
where $0<u<\infty, 0<v<2\pi, \alpha=\mbox{const.}$,
is a parametrization of the cone with $2\alpha$ as the angle of the vertex. In the corresponding coordinate neighborhood, prove that the curve
\[
\mathbf{x}(c\exp(v\sin\alpha \cot \beta), v),\ \ \ c=\mbox{const.}, \beta = \mbox{const.},
\]
intersects the generators of the cone ($v=$ const.) under the constant angle $\beta$.
\end{problem}
\begin{proof}
\begin{align*}
d\mathbf{x} = \left[\begin{array}{ccc}\sin\alpha\cos v & \sin\alpha\sin v & \cos\alpha \\ -u\sin\alpha\sin v & u\sin\alpha\cos v & 0\end{array}\right]
\end{align*}
We have $\mathbf{x}_u \times \mathbf{x}_v = (-, -, u)\neq 0$ whenever $u > 0$.
To show that the angle is $\beta$, define
\begin{align*}
E &:= \langle \mathbf{x}_u, \mathbf{x}_u\rangle = \sin^2\alpha\cos^2 v + \sin^2\alpha\sin^2 v + \cos^2\alpha) = 1\\
F &:= \langle \mathbf{x}_u, \mathbf{x}_v\rangle = -u\sin^2\alpha\cos v\sin v + u\sin^2\alpha\sin v\cos v + 0 = 0\\
G &:= \langle \mathbf{x}_v, \mathbf{x}_v\rangle = u^2\sin^2\alpha\sin^2 v + u^2\sin^2\alpha\cos^2 v + 0 = u^2\sin^2\alpha\\
\langle\langle (a, b), (c, d)\rangle\rangle &:= 
\mat abcdEFG
\end{align*}
to be the first fundamental form of $\mathbf{x}$. Then
\begin{align*}
A &:= \partial(u, v)/\partial u = (1, 0) \\
B &:= \partial(c\exp(v\sin\alpha \cot \beta), v)/\partial v = (c\sin\alpha\cot\beta\exp(v\sin\alpha\cot\beta), 1).
\end{align*}
Fixing the intersection at $(c\exp(v\sin\alpha\cot\beta), v)$, we got
\begin{align*}
\langle\langle A, B\rangle\rangle &= \mat abcdEFG  \\&= \mat 10{c\sin\alpha\cot\beta\exp(v\sin\alpha\cot\beta)}110{c^2\exp(2v\sin\alpha\cot\beta)\sin^2\alpha} \\&= c\sin\alpha\cot\beta\exp(v\sin\alpha\cot\beta)\\
\langle\langle A, A\rangle\rangle &= \mat ababEFG  \\&= \mat 101010{c^2\exp(2v\sin\alpha\cot\beta)\sin^2\alpha} \\&= 1\\
\langle\langle B, B\rangle\rangle &= \mat cdcdEFG  \\&= \matnl {c\sin\alpha\cot\beta\exp(v\sin\alpha\cot\beta)}1{c\sin\alpha\cot\beta\exp(v\sin\alpha\cot\beta)}110{c^2\exp(2v\sin\alpha\cot\beta)\sin^2\alpha} \\&= c^2\sin^2\alpha\cot^2\beta\exp(2v\sin\alpha\cot\beta) + c^2\sin^2\alpha\exp(2v\sin\alpha\cot\beta)\\
&=c^2\sin^2\alpha\csc^2\beta\exp(2v\sin\alpha\cot\beta)\\
\cos\theta &= \frac{\langle\langle A, B\rangle\rangle}{\sqrt{\langle\langle A, A\rangle\rangle\langle\langle B, B\rangle\rangle}} = \frac{c\sin\alpha\cot\beta\exp(v\sin\alpha\cot\beta)}{\left|c\sin\alpha\csc\beta\exp(v\sin\alpha\cot\beta)\right|} = \pm \cos\beta\\
\end{align*}
\end{proof}

\end{document}
