\documentclass[10pt,a4paper]{article}
\usepackage{amsmath, amssymb, amsthm}
%加這個就可以設定字體
\usepackage{fontspec}
\usepackage{xkeyval} %MikTeX 2.9 版本相容有誤, 以此修正
%使用xeCJK,其他的還有CJK或是xCJK
\usepackage{xeCJK}
%設定英文字型,不設的話就會使用預設的字型
%\setmainfont{Times New Roman}
%設定中英文的字型
%字型的設定可以使用系統內的字型,而不用像以前一樣另外安裝
%\setCJKmainfont{文泉驛微米黑}
\setCJKmainfont{WenQuanYi Micro Hei}
\setCJKfamilyfont{lh}{LiHei Pro}
\newcommand{\LiHei}{\CJKfamily{lh}}
%中文自動換行
\XeTeXlinebreaklocale "zh"
%文字的彈性間距
\XeTeXlinebreakskip = 0pt plus 1pt
%設定段落之間的距離
\setlength{\parskip}{0.3cm}
%設定行距
%\linespread{1.5}\selectfont

%先暫時用水泥字型,如果任何人看不下去就改吧XD
\usepackage[T1]{fontenc}
\usepackage{concmath}
%\usepackage{mathptmx} %Times

\newcounter{theProblemCounter}
\newtheorem{problem}[theProblemCounter]{Problem}

\begin{document}
\title{{\fontspec{Copperplate Gothic Bold}Geometry Homework 5}}
%\title{{\fontspec{Copperplate}Geometry Homework 4}}
\author{{\it{B96201044}} {\LiHei 黃上恩}, {\it{B98901182}} {\LiHei 時丕勳}, {\it{K0020100x}} {\LiHei 劉士瑋}}
\date{\today}
\maketitle

\newcommand{\bx}{\mathbb{X}}
%第一題
\setcounter{theProblemCounter}{0}
\begin{problem}[參見 P67 Ex16]
考慮
\[
\begin{array}{ccccc}
\bx: & \mathbb{R}^2 & \to & S^2\setminus \{N\} & , N=(0,0,1) \\
& (u, v) & \mapsto & \left(\frac{2u}{u^2+v^2+1}, \frac{2v}{u^2+v^2+1},\frac{u^2+v^2-1}{u^2+v^2+1}\right) &
\end{array}
\]

\begin{enumerate}
\item[(a)] 檢查這的確是 $S^2\setminus \{N\}$ 的參數式
\item[(b)] 計算 $E, F, G$,$E=G$ 嗎?
\item[(c)] 計算 $\bx_u, \bx_v$
\item[(d)] 若 $W_1, W_2$ 是 $\mathbb{R}^2$ 兩以 $a$ 為起點的向量,說明 $W_1, W_2$ 的夾角 $=d\bx(W_1)$ 與 $d\bx(W_2)$ 的夾角
\end{enumerate}
\end{problem}
\begin{proof}
\end{proof}

%第二題
\setcounter{theProblemCounter}{1}
\begin{problem}[旋轉面]
$\bx(\theta, s)=(a(s)\cos\theta, a(s)\sin\theta, b(s))$,其中 $(a(s), b(s))$ 為長度參數之平面曲線。計算 $E, F, G$ 並討論其 regular 的條件。
\end{problem}
\begin{proof}
\begin{align*}
\bx_{\theta}&=(-a(s)\sin\theta, a(s)\cos\theta,0)\\
\bx_{s}&=(a'(s)\cos\theta, a'(s)\sin\theta, b'(s))\\
\rightarrow E&=a(s)^2\sin^2\theta+a(s)^2\cos^2\theta\\
&=a(s)^2\\
F&=-a(s)a'(s)\sin\theta\cos\theta+a(s)a'(s)\cos\theta\sin\theta\\
&=0\\
G&=a'(s)^2\sin^2\theta+a'(s)^2\cos^2\theta\\
&=a'(s)^2\\
&=1\\
\left|\bx_{\theta}\times\bx_{s}\right|&=\sqrt{EG-F^2}\\
&=\sqrt{a(s)^2}\\
&=\left|a(s)\right|\\
\end{align*}
So $\bx$ is regular iff $a(s)\neq 0$.
\end{proof}

%第三題
\setcounter{theProblemCounter}{2}
\begin{problem}[管面]
設空間曲線 $\gamma(s)$, $s$ 長度參數,$\vec{t}, \vec{n}, \vec{b}$ 為 Frenet frame。令 $\bx_l(s,\theta)=\gamma(s)+l\cos\theta\vec{n}(s) + l\sin\theta\vec{b}(s), l>0$,計算 $E, F, G$ 並討論其 regular 條件。
\end{problem}
\begin{proof}
\end{proof}

%第六題
\setcounter{theProblemCounter}{5}
\begin{problem}[Ex6, p100]
Show that
\[ \mathbf{x}(u, v) = (u\sin\alpha\cos v, u\sin\alpha\sin v, u\cos\alpha)
\]
where $0<u<\infty, 0<v<2\pi, \alpha=\mbox{const.}$,
is a parametrization of the cone with $2\alpha$ as the angle of the vertex. In the corresponding coordinate neighborhood, prove that the curve
\[
\mathbf{x}(c\exp(v\sin\alpha \cot \beta), v),\ \ \ c=\mbox{const.}, \beta = \mbox{const.},
\]
intersects the generators of the cone ($v=$ const.) under the constant angle $\beta$.
\end{problem}
\begin{proof}
\end{proof}

\end{document}
